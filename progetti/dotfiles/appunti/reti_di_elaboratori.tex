\documentclass[a4paper,10pt]{article} %default
\usepackage{geometry}
\usepackage{fontspec}
\usepackage[normalem]{ulem}
\usepackage{color}
\usepackage{listings}
%\usepackage{tikz}
%\usetikzlibrary{arrows.meta}
%\usetikzlibrary{graphs,graphdrawing}
%\usegdlibrary{force}
\geometry{top = 3mm, lmargin=5mm, rmargin=5mm, bottom=3mm}
\pagestyle{empty}
\setlength{\parindent}{0pt}

\lstset{
    basicstyle=\ttfamily,
    numberstyle=\ttfamily,
    numbers=left,
    backgroundcolor=\color[rgb]{0.9,0.9,0.9},
    columns=fullflexible,
    keepspaces=true,
    frame=lr,
    framesep=8pt,
    framerule=0pt,
    xleftmargin=20pt,
    xrightmargin=30pt,
    mathescape
}

\newcommand{\specialcell}[2][c]{%
    \begin{tabular}[#1]{@{}l@{}}#2\end{tabular}}
\newcommand{\dimo}[1]{%
    \smallskip \par \hfill\begin{minipage}{0.97\linewidth}{ \scriptsize {\textbf{\em{Dim.}}} {#1} }\end{minipage} \smallskip \par}
\newcommand{\mtheorem}[1]{%
    {\hspace*{-10pt} \textsc {#1}}}
\newcommand{\malgorithm}[1]{%
    {\bigskip \par \hspace*{-10pt} \underline{\textbf {#1}}}}
\newcommand{\msection}[1]{%
    {\bigskip \par \normalsize \textsc {#1}}\par}
\renewcommand{\b}[1]{%
    {\textbf{#1}}}
\newcommand{\mdef}[1]{%
    {\smallskip\par\begin{tabular}{ll} \textbf{Def.} & \begin{minipage}[c]{0.8\columnwidth}\normalsize  {#1}\end{minipage}\tabularnewline \end{tabular}}\smallskip\par}
\newcommand{\mcomment}[1]{%
    {\hfill \scriptsize {#1}}}
\newcommand{\mprop}[1]{%
    {\smallskip\par\begin{tabular}{ll} \textbf{Prop.} & \begin{minipage}[c]{0.8\columnwidth}\emph  {#1}\end{minipage}\tabularnewline \end{tabular}}\smallskip\par}

\begin{document}

\msection{Dati}
Professore: Paul Wollan e Sergio De agostino.

Programma dettagliato sul sito.


\msection{mer 1/03/2017}

Perdere un secondo nella trasmissione costa 6 miliardi di dollari.
Nella fibra, un bit flippa ogni miliardo. Nel wifi, un bit flippa ogni mille.
All'esame chiede molto molto spesso i ritardi di trasmissione e di propagation.
La perdita di un pacchetto è uguale al servizio postale. Puoi scegliere di mandare cartoline oppure raccoandate, dipende dalla sicurezza che vuoi. Il costo è variabile.


\msection{mar 4/04/2017}

{\huge manca la parte su ipv4}

\b{ipv6}

\emph{datagram format}: ci sta un campo versione, un campo \b{priority}, un campo \b{next-header}...

Inoltre, è stato rimosso il campo \b{checksum}, per ravvelocizzare il tempo di processing a ogni hop; il campo \b{options} è ammesso, ma fuori dall'header, se indicato dal campo \b{next-header}.

\b{ICMP}, internet control message protocol

I pacchetti di ICMP sono usati sia per fare i ping, sia per fare il traceroute, e soprattutto per la segnalazione di errori nel transfer protocol. 
Per esempio, per far funzionare traceroute, invio n datagrammi, con destinazione l'host di destinazione. A ogni datagramma metto un TTL (time to leave) diverso. Il primo a 1, il secondo a 2... In questo modo a ogni router incontrato mi ritorna un ICMP di errore con TTL expired, e ho la traccia dei router attraversati. Tuttavia, ricordarsi che in teoria non è garantito al 100\% che : la strada è sempre la stessa, e che i pacchetti mi ritornano nell'ordine giusto. 

\b{ICMPv6}, icmp fatto per IPv6.  Ci sono messaggi aggiuntivi, come "Packet too big", e altre ottimizzazioni varie.

\b{Transition from Ipv4 to Ipv6} non tutti i router possono essere aggiornati allo stesso tempo. La soluzione adottata fino a oggi è che il datagram IPv6  è contenuto nel payload di un datagram IPv4. In questo modo, se io parlo IPv6, posso anche parlare con uno IPv4. I nuovi router supportano entrambi. Tuttavia, c'è un problema: non è detto che i router attraverso cui passo supportino tutti Ipv6. Questo processo, in cui incapsulo un IPv6 in un IPv4, per farlo passare attraverso router IPv4, si chiama \b{tunneling}



\end{document}
