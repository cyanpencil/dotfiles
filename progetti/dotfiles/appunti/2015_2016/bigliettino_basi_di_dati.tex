\documentclass[a4paper,10pt]{article} %default
\usepackage{geometry}
\usepackage{fontspec}
\usepackage[normalem]{ulem}
\geometry{top = 3mm, lmargin=5mm, rmargin=5mm, bottom=3mm}
\pagestyle{empty}

\newcommand{\specialcell}[2][c]{%
    \begin{tabular}[#1]{@{}l@{}}#2\end{tabular}}
\newcommand{\dimo}[1]{%
    \par \hfill\begin{minipage}{0.99\linewidth}{ \tiny {\textbf{\em{Dim.}}} {#1} }\end{minipage}}
\newcommand{\mtheorem}[1]{%
    {\hspace*{-10pt} \textsc {#1}}}
\newcommand{\malgorithm}[1]{%
    {\textbf {#1}}}
\newcommand{\msection}[1]{%
    {\normalsize \textsc {#1}\\[1ex]}}

\begin{document}

\scriptsize

\msection{Definizioni}
- Dati $n$ domini $D_1, D_2 \ldots D_n$ non necessariamente distinti, una \textbf{relazione} $r$ è un sottoinsieme del prodotto cartesiano $D_1 \times D_2 \times \ldots \times D_n$. \\
- Uno \textbf{schema di relazione} è un insieme di attributi.\\
- Uno \textbf{schema di base di dati} relazionale è un insieme $\{ R_1, R_2 \ldots R_n \}$ di schemi di relazione.\\
- Una \emph{base di dati relazionale} con schema $\{ R_1, R_2 \ldots R_n \}$ è un insieme $\{ r_1, r_2 \ldots r_n \}$ dove $r_i$ è un'\textbf{istanza} di relazione con schema $R_i$.\\
- Un'\textbf{anomalia} è un comportamento inaspettato e indesiderato della base di dati. Può essere di \emph{inserimento}, \emph{cancellazione}, \emph {aggiornamento}, o \emph{ridondanza}. \\
- Una \textbf{dipendenza funzionale} su uno schema di relazione $R$ è una coppia ordinata di sottoinsiemi non vuoti $X, Y, \in R$ e viene denotata come $X \rightarrow Y$. \\
- Un'istanza $r$ di $R$ \textbf{soddisfa} la dipendenza funzionale $X \rightarrow Y$ se \quad $\forall (t1, t2) \in r, \quad t_1[X] = t_2[X] \quad  \Rightarrow \quad t_1[Y] = t_2[Y]$ \\
- Un'istanza $r$ di $R$ è \textbf{legale} se soddisfa tutte le dipendenze di un insieme $F$ su $R$.\\
- La \textbf{chiusura} $F^+$ di un insieme di dipendenze funzionali $F$, è l'insieme di dipendenze funzionali che sono soddisfatte da ogni istanza legale di $R$.\\
- Un sottoinsieme $K$ di $R$ è una \textbf{chiave} per $R$ se $K \rightarrow R \in F^+$, e se $\forall K' \subset K, K' \rightarrow R \notin F^+$.\\
- Un'attributo di $R$ è \textbf{primo} se appartiene a una chiave di $R$.\\
- Un sottoinsieme $X$ di $R$ è una \textbf{superchiave} se contiene una chiave di $R$.\\
- Una dipendenza $X \rightarrow A \in F^+$ si dice \textbf{parziale} se $A$ non è primo e $X$ è contenuto propriamente in una chiave di $R$.\\
- Una dipendenza $X \rightarrow A \in F^+$ si dice \textbf{transitiva} se $A$ non è primo e se $\forall$ chiave $K$ si ha che $X$ non è contenuto propriamente in $K$ e $K - X \neq \emptyset$ \\ 
- Uno schema di relazione $R$ è in 3FN se, $\forall(X \rightarrow A) \in F^+ \; | \; A \notin X$, si ha che $A$ è primo oppure $X$ è una superchiave.\\
\begin{tabular}{@{}llll@{}}
    - \textbf{Assioma della riflessività}: & $Y \subseteq X \subseteq R$ & $\Rightarrow$ & $X \rightarrow Y \in F^A$\\
    - \textbf{Assioma dell'aumento}: & $X \rightarrow Y \in F^A$ & $\Rightarrow$ & $XZ \rightarrow YZ \in F^A, \forall Z \subseteq R$ \\
    - \textbf{Assioma della transitività}: & $X \rightarrow Y \in F^A, Y \rightarrow Z \in F^A$ & $\Rightarrow$ & $X \rightarrow Z \in F^A$\\
\end{tabular}\\
- La \textbf{chiusura di $X$} rispetto a F, $X^+_F$, è definita nel modo seguente: $X^+_F = \{ A \; | \; X \rightarrow A \in F^A \}$\\
- Una \textbf{decomposizione} di $R$ è una famiglia $\rho = \{R_1, R_2 \ldots R_k\}$ di sottoinsiemi di $R$ che \emph{ricopre} $R$ ($\cup_{i=1}^k R_i = R$)\\
- Due insiemi di dipendenze funzionali $F$ e $G$ si dicono \textbf{equivalenti} se $F^+ = G^+$\\
- Una decomposizione $\rho$ \textbf{preserva} $F$ se $F \equiv \cup_{i=1}^k \pi_{R_i}(F)$, dove $\pi_{R_i}(F) = \{X \rightarrow Y \; | \; X \rightarrow Y \in F^+ \wedge XY \subseteq R_i\}$\\
- Una decomposizione $\rho$ di $R$ ha un \textbf{join senza perdita} se per ogni istanza legale di $r$ di $R$ si ha $r = \pi_{R_j}(F) \bowtie \pi_{R_2}(r) \bowtie \ldots \bowtie \pi_{R_k}(r)$\\
- Una \textbf{copertura minimale} di $F$ è un insieme $G$ di dipendenze equivalente a $F$ tale che ogni dipendenza non è ridondante.

\bigskip

\msection{Teoremi}
    \mtheorem {Teorema 1}\\
        \emph{Uno schema $R$ è in 3FN se e solo se non esistono nè dipendene parziali nè dipendenze transitive in $R$.}
        \dimo{\emph{Parte solo se:} Banale.
              \emph{Parte se:} Assurdo - $R$ non è in 3FN; quindi esiste $X \rightarrow A$ tale che $A$ non è primo e $X$ non è superchiave. Ma $X \rightarrow A$ è perforza o parziale o transitiva.}
    \\[1ex]
    \mtheorem {Teorema 2}\\
        \emph{\textbf{Regola dell'unione}}: $X \rightarrow Y \in F^A$ e $X \rightarrow Z \in F^A \quad \Rightarrow \quad X \rightarrow YZ \in F^A$
        \dimo{$XY \rightarrow YZ$ per aumento; $X \rightarrow XZ$ per aumento; $X \rightarrow YZ$ per transitività.}
        \emph{\textbf{Regola della decomposizione}}: $X \rightarrow Y \in F^A$ e $Z \subseteq Y \quad \Rightarrow \quad X \rightarrow Z \in F^A$
        \dimo{$Y \rightarrow Z$ per riflessività; $X \rightarrow Z$ per transitività.}
        \emph{\textbf{Regola della pseudotransitività}}: $X \rightarrow Y \in F^A$ e $WY \rightarrow Z \in F^A \quad \Rightarrow \quad WX \rightarrow Z \in F^A$
        \dimo{$WX \rightarrow WY$ per aumento; $WX \rightarrow Z$ per transitività.}
    \\[1ex]
    \mtheorem {Teorema 3}\\
        \textbf{$F^+ = F^A$}
        \dimo{
            \emph{$F^+ \subseteq F^A$:} Sia $X \rightarrow Y \in F^A$. Dimostriamo $X \rightarrow Y \in F^+$ per induzione su Armstrong. Passo induttivo ($i = 0$): $X \rightarrow Y \in F$, quindi 
            $X \rightarrow Y \in F^+$. Induzione: per ipotesi, al passo $i - 1$, la dipendenza ottenuta è in $F^+$. Ora, con quale assioma abbiamo ottenuto $X \rightarrow Y$?:
            $1)$ Riflessività. $Y \subseteq X$; Se $t_1[X] = t_2[X]$, banalmente $t_1[Y] = t_2[Y]$. 
            $2)$ Aumento. $X \rightarrow Y$ viene da $VZ \rightarrow WZ$. Presi $t_1, t_2$ tali che $t_1[X] = t_2[X]$, allora $t_1[V] = t_2[V]$ e $t_1[Z] = t_2[Z]$. Ma da $V$ si ottiene $t_1[W] = t_2[W]$ per hp 
                 induttiva. E da $W$ e da $Z$ si ottiene $t_1[Y] = t_2[Y]$.
            $3)$ Transitiva. Ho che $X \rightarrow Z$ e $Z \rightarrow Y$. $t_1[X] = t_2[X]$ implica che $t_1[Z] = t_2[Z]$ che implica che $t_1[Y] = t_2[Y]$, per hp induttiva.
            \emph{$F^+ \subseteq F^A$:} Assurdo. Diciamo che esiste una $X \rightarrow Y$ in $F^+$ ma non in $F^A$. Mostriamo che esiste un'istanza legale di $R$ che non soddisfa $X \rightarrow Y$.
            Tabella con due righe: alla prima tutti $1$; nella seconda tutti $1$ fino alla fine di $X^+$, poi tutti $0$. $r$ è legale poichè se $V \rightarrow W$ non fosse legale, succederebbe che $V \subseteq X^+$
            e $W \cap (R - X^+) \neq \emptyset$. Per Lemma 1 però $X \rightarrow V$, e per transitività $X \rightarrow W$, e di nuovo lemma 1 $W \subseteq X^+$, contraddizione. Infine $r$ non soddisfa
            $X \rightarrow Y$. Se lo facesse, $Y \subseteq X^+$ per forza, poichè ci sono solo due possibili tuple e devono avere tutto coincidente. Ma per Lemma 1, $X \rightarrow Y \in F^+$, contraddizione.
        }
    \\[1ex]
    \mtheorem {Teorema 4}\\
        \emph{L'\textbf{Algoritmo 1} calcola correttamente la chiusura di un insieme di attributi $X$ rispetto a un insieme $F$ di dipendenze.}
        \dimo{
            Si dimostrerà che, con $j$ corrispondente a quando l'algoritmo termina, $A \in X^+ \Leftrightarrow A \in Z^j$
            \emph{Parte solo se:} Dimostriamo $\forall i \; Z^i \subseteq X^+$. Base induzione: $Z^0 = X$, $X \subseteq X^+$ per riflessione.
            Passo induttivo: $Z^{i-1} \subseteq X^+$ per hp induttiva; Prendiamo un attributo $A \in Z^i - Z^{i-1}$, che deve essere venuto da $Y \rightarrow V$ tale che $Y \subseteq Z^{i-1}$ e $A \in V$. $X \rightarrow Y$ per Lemma 1 poichè $Y \subseteq X^+$ per hp induttiva; $X \rightarrow V$ per transitività. Quindi $A \in X^+$. Quindi $Z^i \subseteq X^+$
            \emph{Parte se:} Prendiamo attributo $A \in X^+$. $X \rightarrow A$ deve essere soddisfatta in ogni istanza legale di $R$, per il Teorema 3. Prendiamo la stessa istanza $r$ del Teorema 3, con gli $1$ fino a $Z^j$ e poi $0$. $r$ è legale poichè se ci fosse $V \rightarrow W$ non soddisfatta, allora $V \subseteq Z^j$ e $W \cap (R - Z^j) \neq \emptyset$, ma in tal caso si avrebbe $S^j \not \subseteq Z^j$, contraddizione. Ora, siccome deve essere soddisfatta anche $X \rightarrow A$, significa che $A \in Z^j$, poichè $X = Z^0 \subseteq Z^j$, cioè deve stare tra gli uni.
        }
    \\[1ex]
    \mtheorem {Teorema 5}\\
        \emph{L'\textbf{Algoritmo 3} calcola correttamente $X_G^+$, dove $G = \cup_{j=1}^k \pi_{R_j}(F)$}
        \dimo{
            Si dimostrerà che, con $f$ corrispondente a quando l'algoritmo termina, $A \in Z^f \Leftrightarrow A \in X^+_G$. 
            \emph{Parte solo se}: Base induzione: $Z^0 = X, X \subseteq X^+$, quindi $Z^0 \subseteq X^+_G$. Passo induttivo: Sia $A \in Z^i - Z^{i-1}$, allora $\exists$ un indice $j \; | \; A \in (Z^{i-1} \cap R_j)^+_F \cap R_j$. Si ha che $(Z^{i-1} \cap R_j) \rightarrow A \in F^+$ (per teorema 3). Per la definizione di $G$, e poichè $A \in R_j$ e $Z^{i-1} \cap R_j \subseteq R_j$, $(Z^{i-1} \cap R_j) \rightarrow A \in G$. Per hp induttiva, $X \rightarrow (Z^{i-1} \in G^+$, per decomposizione, $X \rightarrow (Z^{i-1} \cap R_j) \in G^+$, per transitività $X \rightarrow A \in G^+$, cioè $A \in X^+_G$, quindi $Z^i \subseteq X^+_G$.
            \emph{Parte se}: Useremo la proposizione che $X \subseteq Y \rightarrow X^+_F \subseteq Y^+_F$. Siccome $X = Z^0 \subseteq Z^f$, dalla proposizione $X^+_G \subseteq (Z^f)^+_G$. Dimostreremo che $Z^f = (Z^f)^+_G$. Eseguiamo l'algoritmo 1, dando come input $Z^f$ e $G$. Se per assurdo $Z^f \neq (Z^f)^+_G$, deve esistere $B \in S^0 \not \in Z'^0$. Siccome si ha che $S^0 = \{ A \; | \; (Y \rightarrow V \in G) \wedge (A \in V) \wedge (Y \subseteq Z'^0) \}$, per la definizione di $G$, deve esistere $j \; | \; B \in \{ A \; | \; (Y \rightarrow V \in F^+) \wedge (A \in V) \wedge (Y \subseteq Z'^0) \wedge (YV \subseteq R_j) \}$. Da $Y \rightarrow V \in F^+$, per il Lemma 1, $V \subseteq Y^+_F$. Inoltre, siccome $Y \subseteq Z'^0 \wedge YV \subseteq R_j$, $Y \subseteq Z'^0 \cap R_j = Z^f \cap R_j$, si ha che $V \subseteq (Z^f \cap R_j)^+_F$, proprio per la proposizione. Infine, poichè $YV \subseteq R_j$, $V \subseteq (Z^f \cap R_j)^+_F \cap R_j$. Ma $B \in V$ e quindi $B \in (Z^f \cap R_j)^+_F \cap R_j$ e quindi $B \in S^f$. E siccome $B \not \in Z^f$, questo significa che $Z^f$ non è il valore finale di $Z$ (contraddizione).
        }
    \\[1ex]
    \mtheorem {Teorema 6}\\
        \emph{$m_p(r) = \pi_{R_1}(r) \bowtie \ldots \bowtie \pi_{R_k}(r)$. Si ha che:}\\
        - $r \subseteq m_p(r)$
        \dimo{
            Sia $t$ una tupla di $r$. $\forall i \in \{ 1 \ldots k \} , t[R_i] \in \pi_{R_i}(r)$, e quindi $t \in m_{\rho}(r)$
        }
        - $\pi_{R_i}(m_p(r)) = \pi_{R_i}(r)$
        \dimo{
            Per la precedente $r \subseteq m_\rho(r)$, quindi $\pi_{R_i}(r) \subseteq \pi_{R_i}(m_\rho(r))$. Mostriamo che $\pi_{R_i}(r) \supseteq \pi_{R_i}(m_\rho(r))$ poichè $\forall$ tupla $t \in m_\rho(r)$ e $\forall i \in \{ 1 \ldots k \}$, $\exists t'\;\; |\;\; t[R_i] = t'[R_i]$.
        }
        - $m_p(m_p(r)) = m_p(r)$
        \dimo{
            Per la precedente $\pi_{R_i}(m_\rho(r)) = \pi_{R_i}(r)$. Allora $m_\rho(m_\rho(r)) = \pi_{R_1}(m_\rho(r)) \bowtie \ldots \bowtie \pi_{R_k}(m_\omega(r)) = \pi_{R_1}(r) \bowtie \ldots \bowtie \pi_{R_k}(r) = m_\rho(r)$.
        }
    \\[1ex]
    \mtheorem {Teorema 7}\\
        \emph{L'\textbf{Algoritmo 4} decide correttamente se $\rho$ ha un join senza perdita.}
        \dimo{
            Bisogna dimostrare che $\rho$ ha un join senza perdita $\Leftrightarrow$ la tabella $r$ ha una riga con tutte $a$.
            \emph{Parte solo se}: La tabella $r$ puù essere interpretata come un'istanza legale di $R$, poichè l'algoritmo termina quando non ci sono più violazioni delle dipendenze in $F$. Inoltre, non viene modificata nessuna $a$. Se per assurdo $\rho$ ha un join senza perdita ma la tabella non ha una riga con tutte $a$, allora $\forall i \in \{ 1 \ldots k \}$, $\pi_{R_i}(r)$ contiene una riga con tutte $a$; allora $m_\rho(r)$ contiene una riga con tutte $a$, e quindi $m_\rho(r) \neq r$ (contraddizione).
        }
    \\[1ex]
    \mtheorem {Teorema 8}\\
        \emph{L'\textbf{Algoritmo 5} permette di calcolare in tempo polinomiale una decomposione $\rho$ tale che ogni schema in $\rho$ è in 3FN e che $\rho$ preserva $F$}
        \dimo{
            \emph{$\rho$ preserva $F$}: Sia $G = \cup_{i=1}^k \phi_{R_i}(F)$. Siccome $\forall X \rightarrow A \in F$ vale che $XA \in \rho$, si ha che $G \supseteq F$, e $G^+ \supseteq F^+$. Inoltre, $G^+ \supseteq F^+$ siccome per definizione $G \supseteq F^+$.
            \emph{Ogni schema in $\rho$ è in 3FN}: Prendiamo $S \in \rho$. Ogni attributo in $S$ fa parte della chiave, e quindi $S$ è in 3FN (Ricorda che $S$ è l'insieme degli attributi non contenuti in nessuna dipendenza funzionale). Se $R \in \rho$ vuol dire che c'è una dipendenza funzionale che coinvolge tutti gli attributi di $R$. Sarà della forma $R - A \rightarrow A$ poichè $F$ è una copertura minimale; e $R - A$ è chiave in $R$. Sia $Y \rightarrow B \in F^+$; se $B = A$ allora $Y = R - A$, poichè $F$ è copertura minimale, e quindi $Y$ è superchiave; se $B \neq A$ allora $B \in R - A$ e quindi $B$ è primo. Se $XA \in \rho$, non esiste $X' \rightarrow A \in F^+ \; | \; X' \supset X$ poichè $F$ è copertura minimale, e quindi $X$ è chiave in $XA$. Sia $Y \rightarrow B \in F^+ \; | \; YB \supseteq XA$; se $B = A$, allora $Y = X$, cioè $Y$ è superchiave, poichè $F$ è copertura minimale; se $B \neq A$, allora $B \in X$ e quindi $B$ è primo.
        }
    \\[1ex]
    \mtheorem {Teorema 9}\\
        \emph{La decomposione $\sigma = \rho \cup \{K\}$, dove $K$ è una chiave, è tale che ogni schema in $\sigma$ è in 3FN, $\sigma$ preserva $F$, e che $\sigma$ ha un join senza perdita.}
        \dimo{
            \emph{$\sigma$ preserva $F$}: Sia $\sigma = \{ R_1 \ldots R_{k+1} \}$ dove $\rho = \{ R_1 \ldots R_k \}$ è ottenuta con Algoritmo 5 e $R_{k+1} = K$. Sia $G' = \cup_{i=1}^{k+1} \{ X \rightarrow Y \in F^+ \; | \; XY \in R_i \}$, e $G = \cup_{i=1}^{k} \{ X \rightarrow Y \in F^+ \; | \; XY \in R_i \}$. Per Teorema 8 $F \equiv G$; $G \subseteq G'$ e quindi $G \subseteq G'^+$. Per definizione, $G' \subseteq F^+$, e $G' \subseteq G^+$ poichè $F^+ = G^+$. Per il lemma 2, siccome $G \subseteq G'$ e $G' \subseteq G^+$, allora $G \equiv G'$.
            \emph{Ogni schema in $\rho$ è in 3FN}: è sufficiente mostrare che $K$ è in 3FN, poichè $\sigma = \rho \cup \{ K \}$. Se $K$ non fosse chiave di $K$, esisterebbe $K' \subset K \; | \; K' \rightarrow K \in F^+$. Ma $K \rightarrow R$ poichè $K$ è chiave di $R$ per definizione, e per transitività $K' \rightarrow R \in F^+$, che contraddice il fatto che $K$ è chiave di $R$.
            \emph{$\sigma$ ha un join senza perdita}: Mostriamo che viene prodotta una tabella con tutte $a$ dall'algoritmo 4. Supponiamo che l'algoritmo esamini le dipendenze $Y_1 \rightarrow A_1 \ldots Y_n \rightarrow A_n$ dove $A_i$ è l'i-esimo attributo esaminato dall'Algoritmo 1 per calcolare $K^+$, mentre $Y_i \subseteq Z^{i-1} = KA_1, \ldots A_{i-1} \subseteq K^+$. Base induzione: Poichè $Y_1 \subseteq Z^0 = K$, sia nella riga dello schema $Y_1A_1$ che in quella di $K$ ci sono tutte $a$ in corrispondenza degli attributi di $Y_1$, e c'è una $a$ in corrispondenza di $A_1$ nella riga $Y_1A_1$. Allora l'algoritmo 4 mette una $a$ nella riga di $K$ in corrispondenza di $A_1$. Induzione: Per hp induttiva, nella riga di $K$ c'è una $a$ $\forall j \; | \; j \leq i - 1$. Poichè $Y_i \subseteq KA_1 \ldots A_{i-1}$, nella riga di $Y_iA_i$ e nella riga di $K$ ci sono tutte $a$ in corrispondenza degli attributi in $Y_i$; inoltre nella riga di $Y_iA_i$ c'è una $a$ in corrispondenza di $A_i$. Allora l'algoritmo 4 mette una $a$ in corrispondenza di $A_i$ nella riga di $K$. Quindi alla fine nella riga di $K$ ci saranno tutte $a$.
        }
    \\[1ex]
    \mtheorem {{Lemma 1}}\\
        - $Y \subseteq X^+ \Leftrightarrow X \rightarrow Y \in F^A$
        \dimo{Sia $Y = A_1 \ldots A_n$; \emph{Parte se:} $\forall i X \rightarrow A_i$ poichè $Y \subseteq X^+$; per unione, $X \rightarrow Y$. 
              \emph{Parte solo se:} $X \rightarrow Y$; per decomposizione $\forall i X \rightarrow A_i$. Quindi $A_1 \ldots A_n \subseteq X^+$ e $Y \subseteq X^+$}
    \\[1ex]
    \mtheorem {{Lemma 2}}\\
        - $F \subseteq G^+ \;\Rightarrow\; F^+ \subseteq G^+$

\bigskip

\tiny
\msection{Algoritmi}
    \begin{tabular}{lll}
        \specialcell[t]{
        \malgorithm {Algoritmo 1}\\
            $Z := X$\\
            $S := \{A \; | \; (Y \rightarrow V \in F) \wedge (A \in V) \wedge (Y \subseteq Z)\};$\\
            while $S \not \subseteq Z$\\
            \indent $Z := Z \cup S;$\\
            \indent $S := \{A \; | \; (Y \rightarrow V \in F) \wedge (A \in V) \wedge (Y \subseteq Z)\};$\\
        }
        &
        \specialcell[t]{
        \malgorithm {Algoritmo 2}\\
            $successo = true;$\\
            for each $X \rightarrow Y \in F$\\
            \indent calcola $X_G^+;$\\
            \indent if $Y \not \subseteq X_G^+$ then $successo = false;$\\
        }
        &
        \specialcell[t]{
        \malgorithm {Algoritmo 3}\\
            $Z := X;$\\
            $S := \emptyset;$\\
            for $j := 1$ to $k$\\
            \indent $S := S \cup [(Z \cap R_j)_F^+ \cap R_j];$\\
            while $S \not \subseteq Z$\\
            \indent $Z := Z \cup S;$\\
            \indent for $j := 1$ to $k$\\
            \indent \indent $S := S \cup [(Z \cap R_j)_F^+ \cap R_j];$\\
        }
    \end{tabular}
    \\[1ex]
    \begin{tabular}{ll}
        \specialcell[t]{
            \malgorithm {Algoritmo 4}\\
                Costruisci una tabella con $|R|$ colonne e $|\rho|$ righe. All'incrocio dell'i-esima riga e \\
                della j-esima colonna, si metta $a_j$ se $A_j \in R_i$, altrimenti $b_{i,j}$\\
                repeat\\
                for each $X \rightarrow Y \in F$\\
                \indent if $\exists\{t1, t2\} \in r \;|\; t_1[X] = t_2[X] \wedge t_1[Y] \neq t_2[Y]$ then\\
                \indent \indent for each $A_j \in Y$\\
                \indent \indent \indent if $t_1[A_j] = "a_j"$ then $t_2[A_j] := t_1[A_j];$\\
                \indent \indent \indent else $t_1[A_j] := t_2[A_j];$\\
                until $r$ ha una riga con tutte $"a"$ oppure $r$ non è cambiato\\
                if $r$ ha una riga con tutte $"a"$, allora $\rho$ ha un join senza perdita.\\
        }
    &
        \specialcell[t]{
            \malgorithm {Algoritmo 5}\\
                $S := \emptyset;$\\
                for each $A \in R$ tale che $A$ non è in nessuna dipendenza in $F$\\
                \indent $S := S\cup\{A\};$\\
                if $S \neq \emptyset$ then\\
                \indent $R := R - S;$\\
                \indent $\rho := \rho \cup \{S\};$\\
                if esiste una dipendenza in $F$ che coinvolge tutti gli attributi in $R$ then\\
                else for each $X \rightarrow A \in F$ do $\rho := \rho \cup \{XA\};$\\
                \indent $\rho := \rho \cup \{R\};$\\
    }
    \end{tabular}

\bigskip



\msection{Anomalie}
    Anomalie: anomalie di inserimento, di cancellazione, di aggiornamento. Ridondanza dei dati.


\bigskip

\msection{Da rivedere}
    Da rivedere la definizinoe della copertura minimale.

\bigskip

\msection{Organizzazione dei File}
    \mtheorem{File Heap}\\
        -Inserimento: 1 accesso in lettura (leggo l'ultimo blocco); 1 accesso scrittura (scrivi il blocco modificato, chiedine uno nuovo al fyle system se pieno)\\
        -Ricerca: sequenziale blocco per blocco. Da $1$ a $num blocchi$ accessi. Costo medio: $(R + 2R \ldots + nR)/N = R/N * n(n + 1) / 2 = n / 2$ con $N$ numero totale di record, $R$ record per blocco.\\
        -Modifica: Ricerca + 1 accesso scrittura.\\
        -Cancellazione: Ricerca + 1 accesso lettura + 2 accessi scrittura (ultimo + modificato)\\
    \mtheorem{File Hash}\\
        -Ricerca: $1/B$ esimo della ricerca su un heap (assumendo una buona funzione hash, come esempio divido i bit della chiave in gruppi lunghi uguali, li sommi in binario e fai il modulo per $B$)\\
    \mtheorem{File isam (Indexed Sequential Access Method)}\\
        -Ci sono file indice e file principali\\
        -Ricerca: si cerca nel $FI$ un record con valore $K'$ che ricopre $K$, cioè che $K' \leq K$. Costo: $\lceil log_2(num blocchi) \rceil + 1$.\\
        -Ricerca per interpolazione: serve una funzione $f$ che presi $K_1, K_2, K_3$, restituisce la frazione dell'intervallo fra $K_2$ e $K_3$ dove sta $K_1$. Costo: $\lceil log_2 log_2(numblocchi) \rceil + 1$.\\
        -Inserimento: costo ricerca + $1$ se sul blocco c'è spazio, altrimenti di più. Se sono pieni sia quello prima che quello dopo, bisogna chiederne uno nuovo e ripartire i record tra vecchio e nuovo.\\
        -Cancellazione: costo ricerca + $1$. Se il record è il primo del blocco, altri accessi necessari (restituzione blocco a file system e modifica $FI$)\\
        -Modifica: costo ricerca + $1$ se la chiave non cambia, altrimenti costo cancellazione + inserimento.\\
    \mtheorem{File B - Tree}\\
        -Ricerca: $h + 1$, con $h$ altezza dell'albero. Massimo valore di $h = log_d(N/e)$, con $2e - 1$ record per blocco $FP$, $2d - 1$ record per blocco $FI$, e $N$ numero record di $FP$.\\
        -Inserimento: costo ricerca + $1$ se c'è spazio, altrimenti costo ricerca + $1$ + $s$, con $s \leq (2h + 1)$ (scrittura blocchi $FI$ in risalita)\\
        -Cancellazione: costo ricerca + $1$ se il blocco rimane pieno almeno a metà. Altrimenti più accessi.\\
        -Modifica: costo ricerca + $1$ se non cambia la chiave, altrimenti cancellazione + inserimento.

\end{document}
