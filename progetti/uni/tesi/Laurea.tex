% !TeX encoding = UTF-8
% !TeX program = pdflatex
% !TeX spellcheck = it_IT

\documentclass[Lau,binding=0.6cm]{sapthesis}

\usepackage{microtype}
\usepackage[italian]{babel}
\usepackage[utf8]{inputenx}

\usepackage{hyperref}
\hypersetup{pdftitle={Sapthesis class example},pdfauthor={Francesco Biccari}}

% Remove in a normal thesis
\usepackage{lipsum}
\usepackage{curve2e}
\definecolor{gray}{gray}{0.4}
\newcommand{\bs}{\textbackslash}

% Commands for the titlepage
\title{Sentiment Analysis on short internet comments}
\author{Luca Di Bartolomeo}
\IDnumber{1710425}
\course{Informatica}
\courseorganizer{Facoltà di Ingegneria dell'informazione, Informatica e Statistica}
\AcademicYear{2017/2018}
\copyyear{2018}
\advisor{Prof. Alessandro Panconesi}
\coadvisor{Dr. ??? ???}
\authoremail{lucadb@gmail.com}

\examdate{22 Luglio 2018}
\examiner{Prof. ??? ???}
\examiner{Prof. ??? ???}
\examiner{Dr. ??? ???}
\versiondate{\today}



\begin{document}

\frontmatter

\maketitle

\dedication{Dedicato a\\ Donald Knuth}

\begin{abstract}
Sommario.
\end{abstract}

\begin{acknowledgments}
Ringraziamenti.
\end{acknowledgments}

\tableofcontents

% Do not use the starred version of the chapter command!
%\chapter{Capitolo non numerato}
%In this manual you can skip the gray text because it is just dummy text.%
%\footnote{This is a footnote.}
%\textcolor{gray}{\lipsum[1-22]}
%\section*{Paragrafo non numerato}
%In this manual you can skip the gray text because it is just dummy text.
%\textcolor{gray}{\lipsum[1-22]}


\Large


\mainmatter

\chapter{Introduzione}

\chapter{Il problema}

Il termine \emph{Sentiment Analysis}, spesso sostiuito anche con \emph{Opinion
Mining}, si riferisce all'uso tecniche quali Natural Language Processing, text
analysis, computational linguistics e biometrics per identificare, estrarre,
quantificare e studiare stati affettivi e 
informazioni soggettive. 

In generale, l'analisi dei sentimenti di un certo documento ha come obiettivo quello di determinare l'atteggiamento dell'autore del 
documento riguardo l'argomento trattato. L'atteggiamento è un termine generico che rappresenta il giudizio o lo stato emotivo o anche l'emozione che l'autore voleva
instillare nel lettore.

Uno dei task fondamentali del sentiment analysis è quello di classificare la
\emph{polarità} di un certo documento, cioè di attribuire al testo un sentimento
positivo o negativo. 
A prima vista può sembrare un compito banale, 
ecco infatti qualche esempio di frase la cui polarità è ovvia agli occhi di qualunque
lettore:

\begin{itemize}
	\item  "La Fiat ha da sempre prodotto le automobili più affidabili al mondo"
	\item  "Provo un profondo disprezzo per chi guidi una decappotabile".
	\item  "Le ibride sono un ottimo compromesso tra comodità e attenzione all'ambiente"
\end{itemize}

		spesso anche gli umani hanno difficoltà a categorizzare frasi del linguaggio comune.
Tuttavia, spesso il linguaggio naturale è ambiguo o superficiale; specialmente
quando il testo in questione è corto, a volte si rivela particolarmente difficile
categorizzare la polarità.
Si noterà come anche il lettore più attento avrà difficoltà a trovare una precisa categoria nei seguenti esempi:

\begin{itemize}
	\item  "Adoro la mia automobile, ma sinceramente non la consiglierei a nessuno dei miei colleghi"
	\item  "Non vedo l'ora di schiantarmi a bordo di questo bolide!" \\(possibilmente sarcastico)
	\item  "Se non costasse cos\'i tanto, l'avrei già comprata."
\end{itemize}

Oltre alle ambiguità lessicali un algoritmo è costretto ad affrontare anche 
difficoltà derivanti dall'uso comune di figure retoriche quali sarcasmo e 
ironia, per non parlare poi di costrutti come doppie negazioni o ellissi del 
soggetto, e infine errori ortografici.


\section{Principali metodi}

Gli approcci esistenti per il sentiment analysis si suddividono principalmente in tre
categorie:

\bigskip

\textbf{Tecniche \emph{knowledge-based}}: classificano il testo basandosi sulla presenza
di parole dal significato non ambiguo, come "felice", "annoiato", "terribile", etc\ldots


\bigskip

\textbf{Metodi statistici}: fanno uso di algoritmi di machine learning per analizzare 
anche relazioni grammaticali tra le parole permettendo una comprensione molto più 
approfondita della frase.

\bigskip

\textbf{Approcci ibridi}: sono una combinazione dei due metodi precedenti e usano 
particolari strutture come ontologie e network semantici. [non sono sicuro che l'ontologia sia una struttura]

In questo documento prenderemo in analisi alcuni tra i più conosciuti metodi statistici.

\section{Valutazione}


La precisione di un certo algoritmo di sentiment analysis si misura in base
a quanto i risultati sono in accordo con il giudizio umano. 
Tuttavia, come fatto vedere in precedenza, il giudizio umano non è sempre affidabile:
secondo alcune ricerche fatte su Amazon Mechanical Turk
\cite{amazon}
%\url{https://mashable.com/2010/04/19/sentiment-analysis/#zaPcdqZO45qX}
gli umani esprimono giudizi concordi solo l'80\% delle volte. 

Come vedremo tra poco, i migliori algoritmi si avvicinano a una precisione poco
superiore al 70\%, che anche se non sembra a prima vista un risultato notevole,
si avvicina molto alle performance umane.

\section{Lo stato dell'arte}

\begin{itemize}

\cite{nbsvm}

\end{itemize}


\chapter{Naive Bayes and Logistic Regression}

\section{Introduzione}
\section{Implementazione}
\section{Risultati}

\chapter{Reti neurali}

\section{Introduzione}
\section{Implementazione}
\section{Risultati}

\chapter{Word embeddings}

\section{Introduzione}
\section{Implementazione}
\section{Risultati}

\chapter{Conclusioni}










\chapter{Style features of \textsf{sapthesis}}
In this chapter I will discuss my stylistic choices of \textsf{sapthesis}.
I will show the page layout geometry and I will describe the page style.
\section{Page layout}

The page is fixed at the dimensions of an A4 paper, therefore you have to print your thesis on A4 paper to obtain the best results. The font dimension is fixed at 11\, pt. The text column and the margins are chosen to fill to the best an A4 paper while keeping a reasonable line length (396\, pt) for a good readability. The text height and the text width are in golden ratio (\textasciitilde 1.6180) as well as the outer and inner margins in a two-side document after binding margin removal. Also the top margin (excluding the header) and bottom margin are in the golden ratio. In Fig.~\ref{layout} a sketch of the \textsf{sapthesis} page layout is shown.

\begin{figure}[h]
\centering
\setlength{\unitlength}{0.27mm}
\begin{picture}(420,297)(-210,0)
\polyline(-210,0)(210,0)(210,297)(-210,297)(-210,0)
\Line(0,0)(0,297)
\put(27.05,37.4){\polygon(0,0)(139.2,0)(139.2,223.8)(0,223.8)}
\put(-27.05,37.4){\polygon(0,0)(-139.2,0)(-139.2,223.8)(0,223.8)}
\put(27.05,268.16){\polygon(0,0)(139.2,0)(139.2,4.22)(0,4.22)}
\put(-27.05,268.16){\polygon(0,0)(-139.2,0)(-139.2,4.22)(0,4.22)}
\end{picture}
\caption{Page layout scheme of \textsf{sapthesis class} using a zero binding margin.}
\label{layout}
\end{figure}


\section{Page style}

The captions have a smaller font respect to the text and the label is in boldface. The appearance of the margin notes has been improved.
They have the same font dimension of the footnotes and are typed in italics.
Moreover I defined a new command to typeset margin note aligned to the left on the right page and vice versa on the left page.
Notice that if a binding margin greater than 1.5\, cm is used, the dimensions of the margin notes become too small and very ugly.
Do not use them in this case.

The mathematical objects, figures and tables are numbered within the chapters (e.g. 1.1, 1.2,\ldots for the first chapter, 2.1, 2.2 for the second one and so on\ldots). See for example the number of this simple equation
\begin{equation}
x_{1,2}=\frac{-b\pm\sqrt{b^2-4ac}}{2a}
\end{equation}


The title page is automatically composed when the \texttt{\bs maketitle} command is given.
The parameters needed for the title page, author, title, etc\ldots , are supplied by dedicated commands explained in the next section.
Two copies of the university logo in \texttt{pdf} format, one for color printing and the other one for black and white printing, are supplied in the \textsf{sapthesis} package. They are shown in Fig.~\ref{fig:largenenough}.

\begin{figure}
\centering
\includegraphics[width=0.7\textwidth]{sapienza-MLred-pos}\\[3ex]
\includegraphics[width=0.7\textwidth]{sapienza-MLblack-pos}
\caption{Logo of the Sapienza -- University of Rome.}
\label{fig:largenenough}
\end{figure}



\section{About figures and tables}

As regards the image formats, please use vector images as much as possible! Use jpg images only for photographs! pdf\LaTeX\ supports the pdf, jpg and png formats.

A very simple table is show in Tab.~\ref{tab:letters}. Remember to typeset
always the table caption above the table. Do not use vertical lines.

\begin{table}
\caption{This is a simple table.}
\label{tab:letters}
\centering
\begin{tabular}{lcc}
\toprule
Letter & Test & Test \\
\midrule
A & C & E \\
B & D & F \\
\bottomrule
\end{tabular}
\end{table}


\section{A section}

In this manual you can skip the gray text because it is just dummy text.

\textcolor{gray}{\lipsum[1-10]}



\section{Another section}

In this manual you can skip the gray text because it is just dummy text.

\textcolor{gray}{\lipsum}


\appendix
\chapter{Special commands provided by \textsf{sapthesis}}

\textsf{Sapthesis} provides some special commands, particularly useful for scientific works. You can use for example the roman shape, instead of the italic, for the imaginary unit (\texttt{\bs iu}) and Napier's number (\texttt{\bs eu}):
\begin{equation}
\eu^{\iu\pi}+1=0
\end{equation}

There are also two commands to speed up the writing of derivatives. In the following example we have used the commands \texttt{\bs der} and \texttt{\bs pder}):
\begin{equation}
\der{f}{x} \qquad \pder[2]{f}{y}
\end{equation}


\textsf{Sapthesis} provides also 4 commands to improve the writing of subscripts, \texttt{\bs rb} and \texttt{\bs tb}, and superscripts, \texttt{\bs rp} and \texttt{\bs tp}. Two of these commands, \texttt{\bs rb} and \texttt{\bs rp}, can be used both in text and in math mode and compose their argument in roman. The other two, \texttt{\bs tb} and \texttt{\bs tp}, can be used only in text mode and compose their argument as are. Here it is an usage example of \texttt{\bs rb} and \texttt{\bs rp}:
\[
a_b \neq a\rb{b}\qquad a^b \neq a\rp{b}
\]
And here it is an usage example of \texttt{\bs tb}: \emph{Cu\tb{It} indicates copper bought in Italy}. And a usage example of \texttt{\bs ts}: \emph{Cher G\tp{le} Napolèon}.


Then several commands for the correct typesetting of unit of measurements are provided. For example the command \texttt{\bs un} typesets its argument in roman and leaves a thin space between the number and the unit: $25\un{m}$, $3.5\un{m/s}$. Other commands are: (\texttt{\bs g}) 45\g, (\texttt{\bs C}) 30\,\C, (\texttt{\bs A}) 12\,\A, (\texttt{\bs micro}) 40\,\micro m, (\texttt{\bs ohm}) 27\,\ohm. 

We have also \texttt{\bs x} as abbreviation of \texttt{\bs times}: \$7 \bs x 10\^{}5\$ gives $7 \x 10^5$. Then \texttt{\bs di} is the differential symbol which automatically insert the correct spacing.
\[
\int x \di x
\]

Finally we have defined the color \textsf{sapred} which is the official color
of Sapienza -- University of Rome. It is defined as RGB(130,36,51). \textcolor{sapred}{This text is written with the color \textsf{sapred}.}

In the following dummy text you can observe the usage of \texttt{\bs mnote} command which typesets fancy margin notes.

\textcolor{gray}{\lipsum}
\marginpar{This is a fancy margin note!}
\textcolor{gray}{\lipsum}

\backmatter
% bibliography
\cleardoublepage
\phantomsection
\bibliographystyle{sapthesis} % BibTeX style
\bibliography{bibliography} % BibTeX database without .bib extension

\begin{thebibliography}{9}

\bibitem{amazon}
\url{https://mashable.com/2010/04/19/sentiment-analysis/#zaPcdqZO45qX}



\bibitem{lamport94}
Leslie Lamport,
\textit{\LaTeX: a document preparation system},
Addison Wesley, Massachusetts,
2nd edition,
1994.

\bibitem{nbsvm}
Baselines and Bigrams: Simple, Good Sentiment and Topic Classification,
Sida Wang and Christopher D. Manning,
2016

\end{thebibliography}

\end{document}
