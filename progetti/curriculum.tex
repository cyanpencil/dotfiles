%----------------------------------------------------------------------------------------
%	PACKAGES AND OTHER DOCUMENT CONFIGURATIONS
%----------------------------------------------------------------------------------------

\documentclass[a4paper,10pt]{article} % Default font size and paper size

\usepackage{fontspec} % For loading fonts
\defaultfontfeatures{Mapping=tex-text}
%\setmainfont[SmallCapsFont = Fontin SmallCaps]{Fontin} % Main document font
%\setmainfont[SmallCapsFont = FontinSmallCaps]{Fontin} % Main document font

\usepackage{geometry}
\geometry{lmargin=50pt, rmargin=50pt, tmargin=50pt, bmargin=50pt}


\usepackage{eurosym}
\usepackage{xunicode,xltxtra,url,parskip} % Formatting packages

\usepackage[usenames,dvipsnames]{xcolor} % Required for specifying custom colors

%\usepackage[big]{layaureo} % Margin formatting of the A4 page, an alternative to layaureo can be \usepackage{fullpage}
% To reduce the height of the top margin uncomment: \addtolength{\voffset}{-1.3cm}

\usepackage{hyperref} % Required for adding links	and customizing them
\definecolor{linkcolour}{rgb}{0,0.2,0.6} % Link color
\hypersetup{colorlinks,breaklinks,urlcolor=linkcolour,linkcolor=linkcolour} % Set link colors throughout the document

\usepackage{titlesec} % Used to customize the \section command
\titleformat{\section}{\Large\scshape\raggedright}{}{0em}{}[\titlerule] % Text formatting of sections
\titlespacing{\section}{0pt}{3pt}{3pt} % Spacing around sections

\begin{document}

\pagestyle{empty} % Removes page numbering

\font\fb=''[cmr10]'' % Change the font of the \LaTeX command under the skills section

\bigskip
\bigskip
%----------------------------------------------------------------------------------------
%	NAME AND CONTACT INFORMATION
%----------------------------------------------------------------------------------------

\par{\centering{\Huge \textsc{Luca Di Bartolomeo}}\bigskip\par} % Your name

\section{Personal Data}

\begin{tabular}{rl}
\textsc{Place and Date of Birth:} & Rome, Italy  | 23 September 1996 \\
\textsc{Address:} & Via Apuania 13, Rome, Italy \\
\textsc{Phone:} & 0039 06 44291462\\
\textsc{Mobile:} & 0039 339 3440236\\
\textsc{email:} & \href{lucadb96@gmail.com}{lucadb96@gmail.com}
\end{tabular}

\bigskip
%----------------------------------------------------------------------------------------
%	EDUCATION
%----------------------------------------------------------------------------------------

\section{Education}
\begin{tabular}{rl}
\textsc{2015 - now} & Bachelor's Degree in \textsc{Computer Science} at \textbf{La Sapienza} university of Rome.\\
&Expected graduation date: 22-28 July 2018\\
\textsc{2010 - 2015} & High school diploma at \textbf{Liceo Scientifico ``Augusto Righi''}, in Rome
\end{tabular}

\bigskip
%----------------------------------------------------------------------------------------
%	SKILLS
%----------------------------------------------------------------------------------------

\section{Skills}
\begin{tabular}{rl}
Programming Languages: & Java, C/C++, Python, GLSL, JavaScript, Processing\\
Tools/Technologies: & Vim, Git, Arduino, Android, Google Cardboard, openGL, Raspberry Pi\\
Operating Systems: & Windows, Linux (Ubuntu, Arch Linux)\\
Spoken Languages: & Italian (Native), English (Fluent), Spanish (Basic)\\
\end{tabular}

\bigskip
%----------------------------------------------------------------------------------------
%	WORK EXPERIENCE
%----------------------------------------------------------------------------------------

\section{Work Experience}
%\begin{tabular}{r|p{11cm}}
    \begin{tabular}{rl}
	\textsc{Feb 2018 - now}  & teaching assistant for the Competitive programming course, aimed at preparing highschool \\
	     & students for the regional contest of the Italian Olympiads in Informatics.  \\
	\textsc{Jun - Sep  2017} & Intern at Spiketrap, focused on sentiment analysis of short internet comments in Python\\
	     & using a variety of machine learning algorithms (Naive Bayes, Logistic regression, Neural nets) \\
        2016 & Speaker at Codemotion Rome 2016, talk: "You Turing Complete me – \\
	     & turing completeness in videogames"\\
        %\multicolumn{2}{c}{} \\
        2015 & Coding teacher at Codemotion Kids \\
    \end{tabular}

\bigskip

%----------------------------------------------------------------------------------------
%	COMPETITION EXPERIENCE
%----------------------------------------------------------------------------------------

\section{Project / Competition Experience (Solo)}
\begin{tabular}{rl}
2018 &  \textbf{Raymarching distance fields:} rendered a procedurally generated volcanic archipelagus \\
     & entirely made in the fragment shader in GLSL for the final project of the Computer \\
     %& Graphics course. \href{https://drive.google.com/open?id=1U4zynGm8o4i8VodnaWTOjgoayj9rhega}{Short pdf description.} \\
     & Graphics course.  \\
2016 &  \textbf{Voronoi stippling}: I wrote a program in Python to emulate the stippling painting \\
     & technique using weighted voronois. I was supervised by my Programming Fundamentals \\
     & professor, but it was an extracurricular activity. \\
2015 &  \textbf{Italian Olympiads in Mathematics}: I got the bronze medal in the national finals. \\
     & It was an high school individual competition, which revolved around solving both \\
     & numerical and demonstration problems. \\
2014 &  \textbf{Musical Floppies}: developed for my high school thesis, I connected 8 old floppy \\
     & drives with an Arduino, and by moving their stepper motors at precise frequences \\
     & I was able to play simple tunes with the noise they produced.\\
2014 &  \textbf{Italian Olympiads in Informatics}: I got the 12th place in the national finals \\
     & and got a silver medal. It was an high school competition that revolved around \\
     & the solution and C++ implementation of algorithmical problems. I was also \\
     & selected for the IOI training camps. \\
\end{tabular}

%----------------------------------------------------------------------------------------
%	COMPETITION EXPERIENCE
%----------------------------------------------------------------------------------------

\section{Project / Competition Experience (Team)}
\begin{tabular}{rl}
2018 &  \textbf{Google Hashcode:} partecipated in the Google Hashcode (March 2018) in a team of 4,\\
     & an international programming competition revolving around optimisation problems. We got \\
     & the first place in Italy (out of 300 Italian teams), 85th place in the world (out of 5000 teams). \\
2017 &  \textbf{SWERC 2017}: I partecipated as a member of the "Sapienza Red" team, held in Paris\\
     & in November 2017.\\
2016 &  \textbf{SWERC 2016}: I partecipated as a member of the "Sapienza" team, held in Porto\\
     & in November 2016.\\
2015 &  \textbf{Exhibitor at Codemotion Rome}: I was the main coder of "Pico", a Google Cardboard \\
     & flight simulator for Android in which the player controlled the wings of the plane \\
     & by holding two Wii controllers (connected via bluetooth to the Android phone) in \\
     & his hands and tilting them accordignly. The game got featured and exhibited at \\
     & Codemotion Rome 2015.\\
2015 &  \textbf{Global Game Jam Rome}: I won first prize in the "Graphics" category and second prize \\
     & for "Gameplay" category. Partecipated together with my sister. Our game, "Poopfest",\\
     & featured handmade watercoulour drawings.\\
2013-2015 & \textbf{Italian Olympiads in Mathematics}: I partecipated three times in the national finals\\
     & of the team competition of the Italian Olympiads. We got two medals: a bronze medal \\
     & in 2013 (for the 3rd place), and a silver medal in 2015 (for the 2nd place). In 2013\\
     & and 2014, I was only a member of the team, while in 2015 I was the captain.\\
2013-2015 & \textbf{Ludum Dare Game Jam}: I partecipated five times in this team competition in which you have\\
     & to code a game in 72 hours. In all of those five times I partecipated with my sister;\\
     & I was the main coder, while she was both the artist and the secondary coder. We got \\
     & occasional help from friends for the music. All of our games where made in Java, using \\
     & the Processing library. One of our games got the first place in the "Most Funny" \\
     & category, and 6th place in the "Overall" category (the number of games sent every \\
     & competition floats around a thousand).\\
2013 & \textbf{Cleanweb Hackaton}: I won the first prize by developing a prototype of an application\\
     & to help people visualize if they reduced their electricity consumption in their houses.\\
\end{tabular}


%----------------------------------------------------------------------------------------
%	GRADE TABLES
%----------------------------------------------------------------------------------------

\bigskip
\bigskip

\section{Exams Taken}
\begin{center}
\begin{tabular}{llc}
\multicolumn{1}{c}{\textsc{Exam}} & \multicolumn{1}{c}{\textsc{Grade}} &\textsc{Credit Hrs}\\
Calcolo differenziale (Differential Calculus): & 30 / 30, with honors & 6 \\
Calcolo integrale (Integral Calculus): & 30 / 30 & 6 \\
Progettazione di sistemi digitali (Design of digital systems): & 30 / 30, with honors & 6 \\
Architettura degli elaboratori (Architecture of digital systems): & 30 / 30 & 6 \\
Introduzione agli algoritmi (Introduction to algorithms): & 30 / 30, with honors & 6 \\
Metodi mat. per l'informatica (Math. methods for informatics): & 30 / 30, with honors & 6 \\
Fondamenti di programmazione (Programming fundamentals): & 30 / 30, with honors & 9 \\
Metodologia di programmazione (Programming metodologies): & 30 / 30, with honors & 9 \\
Calcolo delle probabilit\`a (Probability I) & 30 / 30, with honors & 9 \\
Algebra (Geometry I) & 30 / 30 & 9 \\
Progettazione di algoritmi (Algorithm design)  & 30 / 30, with honors & 9 \\
Reti di elaboratori (Computer networking)  & 30 / 30, with honors & 9 \\
Basi di dati (Database) & 27 / 30 & 12 \\
Sistemi operativi (Operative systems) & 30 / 30, with honors & 12 \\
Automi, calcolabilit\`a e complessit\`a (Automatas, calculability and complexity) & 29 / 30 & 6 \\
Computer Graphics & 30 / 30, with honors & 6 \\
Analisi Vettoriale (Vectorial Calculus) & 28 / 30 & 9 \\
Sistemi di basi di dati (Database systems) & 30 / 30 & 6 \\
\end{tabular}
\end{center}
I was selected to take part in the "Excellence Path", a special set of courses aimed at the best students of the faculty, which started in February 2017. I completed it in February 2018.


\bigskip

\section{Hobbies and Interests}
During my spare time I enjoy playing chess, listening to music (in particular when programming), and going out with friends.
I am not a bookworm, but if I stumble upon a book I like, I can't go to sleep unless I've finished it.
\end{document}
