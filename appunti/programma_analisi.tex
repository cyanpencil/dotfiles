%\documentclass[a4paper,10pt]{article} %default
\usepackage{geometry}
\usepackage{fontspec}
\usepackage[normalem]{ulem}
\usepackage{color}
\usepackage{listings}
\usepackage{multicol}
\usepackage{amsmath}
\usepackage{amssymb}
\usepackage{mathtools}
\usepackage{xfrac}
\usepackage{multirow}
\usepackage{abraces}
\geometry{top = 3mm, lmargin=10mm, rmargin=10mm, bottom=3mm}
\pagestyle{empty}
\setlength{\parindent}{0pt}
\setlength{\multicolsep}{0.5em}
\setlength{\columnsep}{0.7cm}


\lstset{
    basicstyle=\ttfamily,
    numberstyle=\ttfamily,
    numbers=left,
    backgroundcolor=\color[rgb]{0.9,0.9,0.9},
    columns=fullflexible,
    keepspaces=true,
    frame=lr,
    framesep=8pt,
    framerule=0pt,
    xleftmargin=20pt,
    xrightmargin=30pt,
    mathescape
}

\newcommand{\specialcell}[2][c]{%
    \begin{tabular}[#1]{@{}l@{}}#2\end{tabular}}
\newcommand{\dimo}[1]{%
    \smallbreak \par \hfill\begin{minipage}{0.92\linewidth}{ \scriptsize {\textbf{\em{Dim.}}} {#1} }\end{minipage} \smallskip \par}
\newcommand{\mdim}[1]{%
    \smallbreak \par \hfill\begin{minipage}{0.95\linewidth}{ \scriptsize {\textbf{\em{Dim.}}} {#1} }\end{minipage} \smallskip \par}
\newcommand{\mtheorem}[1]{%
    {\hspace*{-10pt} \textsc {#1}}}
\newcommand{\malgorithm}[1]{%
    {\bigbreak \par \hspace*{4pt} \underline{\textbf {#1}}}}
\newcommand{\msection}[1]{%
    {\newpage\bigbreak \bigbreak \par \hfil \huge \textsc {#1}}\par}
\renewcommand{\b}[1]{%
    {\textbf{#1}}}
\renewcommand{\t}[1]{%
    {\texttt{#1}}}
\renewcommand{\v}[1]{%
    {\underline{#1}}}
\newcommand{\mdef}[1]{%
    {\smallbreak\par\begin{tabular}{ll} \textbf{Def.$\;\;$} & \begin{minipage}[t]{0.80\columnwidth}\normalsize  {#1}\end{minipage}\tabularnewline \end{tabular}}\smallskip\par}
\newcommand{\ldef}[1]{%
    {\smallbreak\par\normalsize\textbf{\underline{Def.}} {#1} \smallbreak}}
\newcommand{\lprop}[1]{%
    {\smallbreak\par\normalsize\textbf{\underline{Prop.}} {#1} \smallbreak}}
\newcommand{\ltheorem}[1]{%
    {\smallbreak\par\normalsize\textbf{\underline{Th.}} {#1} \smallbreak\par}}
\newcommand{\ldim}[1]{%
    {\smallbreak\par\scriptsize\emph{\textbf{Dimostrazione}} {#1} \par}}
\newcommand{\loss}[1]{%
    {\smallbreak\par\scriptsize\emph{\textbf{Osservazione}} {#1} \par}}
\newcommand{\mcomment}[1]{%
    {\hfill \scriptsize {#1}}}
\newcommand{\mprop}[1]{%
    {\smallbreak\par\begin{tabular}{ll} \textbf{Prop.} & \begin{minipage}[t]{0.8\columnwidth}\emph  {#1}\end{minipage}\tabularnewline \end{tabular}}\smallskip\par}
\newcommand{\shortoverbrace}[2]{\aoverbrace[U]{#1}^{\mathclap{#2}}}%

\makeatletter
\g@addto@macro\normalsize{%
  \setlength{\abovedisplayskip}{3.0pt plus 2.0pt minus 5.0pt}%
  \setlength{\belowdisplayskip}{3.0pt plus 2.0pt minus 5.0pt}%
}
\makeatother

\documentclass[a4paper,10pt]{article} %default
\usepackage{geometry}
\usepackage{fontspec}
\usepackage[normalem]{ulem}
\usepackage{multicol}
\usepackage{multirow}
\usepackage{abraces}
\geometry{top = 3mm, lmargin=8mm, rmargin=8mm, bottom=3mm}
\pagestyle{empty}
\setlength{\parindent}{0pt}
\setlength{\multicolsep}{0.5em}
\setlength{\columnsep}{0.7cm}



\renewcommand*\sfdefault{lcmss}
\renewcommand*\familydefault{\sfdefault} %% Only if the base font of the document is to be sans serif
\usepackage[T1]{fontenc}

%\renewcommand{\bigbreak}{\smallbreak}

\begin{document}

\small

\begin{multicols}{2}

$R^n$ come spazio vettoriale euclideo: modulo di un vettore, prodotto scalare, distanza, disuguaglianza di Cauchy-Schwartz. $R^n$ come spazio topologico: punti interni/esterni, punti di frontiera, punti di accumulazione, insiemi aperti, insiemi chiusi, insiemi compatti (per successioni) e Teorema di Heine-Borel (senza dim). Successioni in $R^n$ e limiti. Teorema di Bolzano-Weierstrass in $R^n$ (senza dim).

Successioni di funzioni: convergenza puntuale e uniforme, Teorema sulla continuità del limite, Teorema di passaggio al limite sotto il segno di integrale, Teorema di passaggio al limite sotto il segno di derivata. 

Norma e prodotto scalare in spazi vettoriali; spazi metrici; successioni convergenti e successioni di Cauchy in spazi metrici; spazi metrici completi e spazi di Banach. Lo spazio delle funzioni continue con la norma del sup è uno spazio metrico completo. Contrazioni e Teorema di Banach-Caccioppoli.


\bigbreak
\hrule
\bigbreak





Serie di funzioni; convergenza puntuale, assoluta, uniforma e totale. Criterio di Cauchy per la convergenza e per la convergenza uniforme. Criterio della convergenza totale. Continuità della somma, integrazione per serie, derivazione per serie.
Serie di potenze: raggio di convergenza, criterio della radice e del rapporto (senza dim). Teorema di Abel (senza dim). Raggio di convergenza della serie derivata.



\bigbreak
\hrule
\bigbreak



Serie di Taylor. Funzioni derivabili infinite volte e funzioni analitiche.

Esercizi su serie di funzioni e serie di potenze.

Curve: curve semplici, chiuse, regolari e curve regolari a tratti. Vettore e versore tangente. Sostegno di una curva. Definizione di lunghezza di una curva. Calcolo della lunghezza di una curva regolare a tratti (senza dim.). Curve cartesiane. Curve polari.



\bigbreak
\hrule
\bigbreak



Due curve equivalenti hanno la stessa lunghezza. Ascissa curvilinea.

Funzioni da Rn a Rm. Definizione di limite e di funzione continua. Una funzione vettoriale è continua se e soltanto se sono continue le componenti scalari che la compongono.

Funzioni da Rn in R. Insieme di definizione. Limiti per funzioni di due variabili. Coordinate polari.

Definizione di derivate parziali, derivate direzionali e vettore gradiente. Una funzione derivabile non è necessariamente continua.




\bigbreak
\hrule
\bigbreak




Differenziabilità, piano tangente, derivate direzionali per funzioni differenziabili, teorema differenziale totale. Esempi.

Funzioni vettoriali di più variabili derivabili,differenziabili, di classe $C^1$. Matrice Jacobiana e differenziabilità di funzioni composte. Il gradiente è ortogonale agli insiemi di livello.

Derivate successive, matrice Hessiana, Teorema di Schwartz, funzioni di classe $C^k$. Teorema di Lagrange, sviluppo di Taylor al secondo ordine con resto Lagrange, sviluppi con resto Peano.



\bigbreak
\hrule
\bigbreak



Insiemi connessi/connessi per archi. Funzioni con gradiente nullo in un aperto connesso sono costanti. Massimi e minimi relativi. Punti critici. Punti di sella. Condizione necessaria di estremalità del primo ordine (Teorema di Fermat). Richiami sulle forme quadratiche. Condizione necessaria di estremalità del primo ordine. Condizioni necessarie e condizioni sufficienti del secondo ordine per l'estremalità. Massimi e minimi assoluti. Teorema di Weierstrass (senza dimostrazione). Teorema dei valori intermedi. Ricerca di massimi e minimi assoluti in domini chiusi e limitati tramite parametrizzazione del bordo.




\bigbreak
\hrule
\bigbreak





Funzioni convesse. Criterio di convessità per funzioni differenziabili e per funzioni di classe $C^2$ (senza dim.).

Teorema delle funzioni implicite (Dini) per funzioni di due variabili (funzioni definite implicitamente da F(x,y)=0). Derivazione della funzione implicita. Formula della derivata seconda della funzione implicita. Teorema del Dini scalare in tre variabili (funzioni definite implicitamente da F(x,y,z)=0) (senza dim.).




\bigbreak
\hrule
\bigbreak




Esonero.

Teorema del Dini per sistemi di due equazioni in tre e quattro variabili (senza dim.). Metodo dei moltiplicatori di Lagrange con un vincolo (dim per n=2), con più vincoli (senza dim.).



\bigbreak
\hrule
\bigbreak




Ricerca del massimo e minimo assoluto per funzioni continue su insiemi compatti.

Integrali curvilinei di prima specie. Massa, centro di massa e momento di inerzia di un filo. Lavoro di un campo vettoriale lungo una curva (integrali di seconda specie). Campi conservativi.

Forme differenziali lineari: [[[definizione]]]. Integrale curvilineo di una forma differenziale. Forme differenziali esatte. Caratterizzazione forme esatte. Forme differenziali chiuse. Le forme esatte sono chiuse. Definizione di rotore e campi vettoriali irrotazionali. Parallelismo tra forme differenziali e campi vettoriali. Ricerca di potenziali.





\bigbreak
\hrule
\bigbreak




Aperti semplicemente connessi in $R^2$ e in $R^3$. Teorema di Poincarè. Forme chiuse in aperti del piano con una lacuna.





\bigbreak
\hrule
\bigbreak




Integrali impropri (secondo Riemann) in R: definizioni e esempi, Teorema del confronto (senza dim.) e del confronto asintotico (senza dim.), una funzione non negativa e integrabile in senso improprio è sommabile secondo Lebesgue e gli integrali coincidono (senza dim.), funzioni assolutamente integrabili in senso improprio.

Teoremi di passaggio al limite: Teorema convergenza monotona (senza dim.), Corollario per serie a termini non negativi, Teorema della convergenza dominata (senza dim.).

Calcolo di integrali doppi: Teorema di Fubini (senza dim.) e Teorema di Tonelli (senza dim.), formule di riduzione in domini normali, area, baricentro e momento di inerzia di domini piani, calcolo di integrali doppi.





\bigbreak
\hrule
\bigbreak



Cambiamento di coordinate negli integrali doppi, coordinate polari.

Calcolo di integrali tripli, formule di riduzione, integrazione per fili e per strati, cambiamento di coordinate negli integrali tripli, coordinate cilindriche, coordinate sferiche, solidi di rotazione, Teorema di Guldino sul volume di un solido di rotazione.

Superfici regolari in $R^3$.





\bigbreak
\hrule
\bigbreak








Piano tangente, versore normale. Area di una superficie, integrali di superficie. Baricentro e momento di inerzia di una superficie.

Teorema di derivazione sotto il segno di integrale (senza dim.), formule di Gauss-Green nel piano, calcolo di aree mediante integrali curvilinei, Teorema di Stokes (o del rotore) nel piano, dimostrazione Teorema di Poincaré, condizione sufficiente affinché un campo irrotazionale in un dominio piano con una lacuna sia conservativo, teorema della divergenza nel piano.

Superfici orientabili, flusso di un campo vettoriale attraverso una superficie. Teorema della divergenza (senza dim.), teorema di Stokes nello spazio (senza dim.)







\bigbreak
\hrule
\bigbreak


Area superfici di rotazione.

Equazioni differenziali di ordine n e riduzione ad un sistema del primo ordine. Equazioni in forma normale e equazioni autonome. Problema di Cauchy. Teorema di Cauchy di esistenza e unicità locale della soluzione del problema di Cauchy. Teorema di Peano di esistenza locale della soluzione del  problema di Cauchy (senza dim.). 

Integrazione di equazioni differenziali lineari del primo ordine,  a variabili separabili,  di tipo omogeneo y’(t)=f(y/t), di Bernoulli.

Soluzione massimale, teoremi di esistenza e unicità globale (senza dim.), teorema dell'asintoto (senza dim.). Analisi qualitativa delle soluzioni di un'equazione differenziale del primo ordine. Equazione logistica.






\bigbreak
\hrule
\bigbreak



Integrale generale di un'equazione differenziale lineare di ordine n. Metodo di somiglianza e metodo di variazione delle costanti per la ricerca di una soluzione particolare di una equazione differenziale lineare non omogenea. Equazioni differenziali di Eulero.

\end{multicols}
\end{document}
