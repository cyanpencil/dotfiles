
\documentclass[a4paper,10pt]{article} %default
\usepackage{geometry}
\usepackage{fontspec}
\usepackage[normalem]{ulem}
\usepackage{color}
\usepackage{listings}
\usepackage{multicol}
\usepackage{amsmath}
\usepackage{amssymb}
\usepackage{mathtools}
\usepackage{xfrac}
\usepackage{multirow}
\usepackage{abraces}
\geometry{top = 3mm, lmargin=10mm, rmargin=10mm, bottom=3mm}
\pagestyle{empty}
\setlength{\parindent}{0pt}
\setlength{\multicolsep}{0.5em}
\setlength{\columnsep}{0.7cm}


\lstset{
    basicstyle=\ttfamily,
    numberstyle=\ttfamily,
    numbers=left,
    backgroundcolor=\color[rgb]{0.9,0.9,0.9},
    columns=fullflexible,
    keepspaces=true,
    frame=lr,
    framesep=8pt,
    framerule=0pt,
    xleftmargin=20pt,
    xrightmargin=30pt,
    mathescape
}

\newcommand{\specialcell}[2][c]{%
    \begin{tabular}[#1]{@{}l@{}}#2\end{tabular}}
\newcommand{\dimo}[1]{%
    \smallbreak \par \hfill\begin{minipage}{0.92\linewidth}{ \scriptsize {\textbf{\em{Dim.}}} {#1} }\end{minipage} \smallskip \par}
\newcommand{\mdim}[1]{%
    \smallbreak \par \hfill\begin{minipage}{0.95\linewidth}{ \scriptsize {\textbf{\em{Dim.}}} {#1} }\end{minipage} \smallskip \par}
\newcommand{\mtheorem}[1]{%
    {\hspace*{-10pt} \textsc {#1}}}
\newcommand{\malgorithm}[1]{%
    {\bigbreak \par \hspace*{4pt} \underline{\textbf {#1}}}}
\newcommand{\msection}[1]{%
    {\newpage\bigbreak \bigbreak \par \hfil \huge \textsc {#1}}\par}
\renewcommand{\b}[1]{%
    {\textbf{#1}}}
\renewcommand{\t}[1]{%
    {\texttt{#1}}}
\renewcommand{\v}[1]{%
    {\underline{#1}}}
\newcommand{\mdef}[1]{%
    {\smallbreak\par\begin{tabular}{ll} \textbf{Def.$\;\;$} & \begin{minipage}[t]{0.80\columnwidth}\normalsize  {#1}\end{minipage}\tabularnewline \end{tabular}}\smallskip\par}
\newcommand{\ldef}[1]{%
    {\smallbreak\par\normalsize\textbf{\underline{Def.}} {#1} \smallbreak}}
\newcommand{\lprop}[1]{%
    {\smallbreak\par\normalsize\textbf{\underline{Prop.}} {#1} \smallbreak}}
\newcommand{\ltheorem}[1]{%
    {\smallbreak\par\normalsize\textbf{\underline{Th.}} {#1} \smallbreak\par}}
\newcommand{\ldim}[1]{%
    {\smallbreak\par\scriptsize\emph{\textbf{Dimostrazione}} {#1} \par}}
\newcommand{\loss}[1]{%
    {\smallbreak\par\scriptsize\emph{\textbf{Osservazione}} {#1} \par}}
\newcommand{\mcomment}[1]{%
    {\hfill \scriptsize {#1}}}
\newcommand{\mprop}[1]{%
    {\smallbreak\par\begin{tabular}{ll} \textbf{Prop.} & \begin{minipage}[t]{0.8\columnwidth}\emph  {#1}\end{minipage}\tabularnewline \end{tabular}}\smallskip\par}
\newcommand{\shortoverbrace}[2]{\aoverbrace[U]{#1}^{\mathclap{#2}}}%

\makeatletter
\g@addto@macro\normalsize{%
  \setlength{\abovedisplayskip}{3.0pt plus 2.0pt minus 5.0pt}%
  \setlength{\belowdisplayskip}{3.0pt plus 2.0pt minus 5.0pt}%
}
\makeatother


\begin{document}



\msection{Analisi Vettoriale 2017}
%\center{Ultima revisione: \today}

\begin{multicols}{2} 

\ldef{
Lo spazio di tutte le $n-uple$ di numeri reali forma uno \b{spazio vettoriale} di dimensione $n$ su $\mathbb{R}$, indicato con $\mathbb{R}^n$. 
Su esso sono definite le operazioni di somma e prodotto: 
\begin{align}
    \b{x} + \b{y} =&\; (x_1 + y_1, x_2 + y_2, \dots) \\
    \emph{a}\b{x} =&\; (ax_1, ax_2, \dots) 
\end{align}
}

\ldef{Il \b{prodotto scalare} si definisce in questo modo:
Ha la proprieta' simmetrica, ed \`{e} lineare rispetto al primo termine: 
\begin{align}
    \langle \b{x, y} \rangle &= \langle \b{y, x} \rangle \\
    \langle \b{x + a, y} \rangle &= \langle \b{x, y} \rangle + \langle \b{a, y} \rangle 
\end{align}
}


\ldef{ Una \b{norma} \`{e} una funzione che assegna ad ogni vettore dello spazio vettoriale, tranne lo zero, una lunghezza positiva. Segue le seguenti proprieta':
    \begin{align}
    ||x||  &\geq  0 \\
    ||x||  &=  0  \Leftrightarrow x = 0 \\
    ||\lambda x|| &=  |\lambda| ||x|| \\
    ||x + y|| &\leq  ||x|| + ||y|| 
    \end{align}
    Alcune norme esemplari sono la norma 1: $\displaystyle ||x||_1 = \sum |x_i|$ \par
    e la norma 2 (euclidea): $\displaystyle ||x||_2 = \sqrt{\sum x_i^2}$
    \loss{
        Due norme $|| \cdot ||_1,|| \cdot ||_2$  si dicono equivalenti se $ \exists c_1, c_2$ tali che $c||x||_1 \leq ||x||_2 \leq C||x||_1, \; \forall x \in V$
    }
    \loss{
        In $\mathbb{R}^n$, tutte le norme sono equivalenti.
    }
}


\ldef{ La \b{distanza} \`{e} una qualsiasi funzione $d: X \times X \rightarrow \mathbb{R}$ che soddisfa le seguenti proprieta':
    \begin{align}
        d(x,y) &\geq 0 \\
        d(x,y) &= 0 \Leftrightarrow x = y \\
        d(x,y) &= d(y,x) \\
        d(x,y) &\leq d(x,z) + d(z,y) 
    \end{align}
    In realta' basta che sono verificate la seconda e la quarta per verificare anche la prima e la terza. 
    Se una distanza segue queste proprieta':
    \begin{align}
        d(x,y) &= d(x+a, y+a) \\
        d(\lambda x, \lambda y) &= |\lambda| d(x,y) 
    \end{align}
    Allora la funzione $||x|| := d(x, 0)$ \`{e} una norma.
    \loss{
        La norma euclidea induce una distanza: $d(x,y) = || \, x - y \, ||_2$
    }
}


\ldef{Uno \b{spazio metrico} \`{e} un insieme di elementi detti punti, nel quale \`{e} definita una funzione distanza, detta anche metrica.
}

\ldef{La disugualianza di \b{Cauchy-Schwartz} dice che il valore assoluto del prodotto scalare di due elementi \`{e} minore o uguale al prodotto delle loro norme:
    $|\langle \b{x,y} \rangle| \leq ||\b{x}|| \cdot ||\b{y}||$
}



\ldef{La successione ${f_n(x)}$ \b{converge} per $x \in E$ alla funzione $f(x)$ se $\forall x_0 \in E$ la successione numerica ${f_n(x_0)}$ converge a $f(x_0)$, cio\`{e}:
    \begin{equation}
        |f_n(x_0) - f(x_0)| < \epsilon \quad \forall \epsilon > 0, \forall x_0 \in E
    \end{equation}
}

\ldef{La successione ${f_n(x)}$\b{converge uniformemente} alla funzione $f(x)$ se $\forall \epsilon > 0$ esiste un'unica soglia $n_\epsilon$ valida per tutti i punti $x_0$, cio\`{e}:
    \begin{gather}
        |f_n(x_0) - f(x_0)| < \epsilon \quad \forall \epsilon > 0 , \forall x_0 \in E, \forall n > n_\epsilon \\
         \text{oppure: } \max_{x\in E} |f_n(x) - f(x)| < \epsilon
    \end{gather}
}

\ltheorem{Il teorema di \b{Bolzano-Weierstrass} afferma che in uno spazio euclideo finito dimensionale $\mathbb{R}^n$ ogni successione reale limitata ammette almeno una sottosuccessione convergente.
}

\ltheorem{Il teorema della continuita' del limite afferma che il limite $f(x)$ di una successione ${f(x)}$ di funzioni continue uniformemente convergenti in un intervallo $I$ \`{e} una funzione continua in $I$.
    \ldim{
        Prendiamo due punti $x_1 \simeq x_2$, e poich\`{e} stiamo parlando di funzioni continue, vale che $f(x_1) \simeq f(x_2)$. \\
        Per la proprieta' triangolare si ha che:
        $$|f(x_1) - f(x_2)| \leq |f(x_1) - f_n(x_2)| + |f_n(x_1) - f_n(x_2)| + +f_n(x_2) - f(x_2)|$$
        Siccome il primo e il terzo modulo a secondo membro sono minori di $\epsilon$:
        $$|f(x_1) - f(x_2)| \leq 2\epsilon + |f_n(x_1) - f_n(x_2)|$$
        Quindi anche la funzione limite $f(x)$ possiede il requisito tipico delle funzioni continue di prendere \emph{valori vicini su punti vicini}
    }
}

\ltheorem{Passaggio al limite sotto il segno di integrale:\\
    Sia ${f_n}$ una successione di funzioni continue su $[a,b]$ tali che $f_n \rightrightarrows f$ (uniformemente), allora:
    $$
        \lim_{n \rightarrow \infty} \int_E f_n(x)dx = \int_E \lim_{n \rightarrow \infty} f_n(x)dx 
    $$
    \ldim{
        Bisogna dimostrare che $ \forall \epsilon > 0, \quad \exists n_\epsilon$ tale che::
        $$
            \left| \int_a^b f_n(x)dx - \int_a^b f(x)dx \right| < \epsilon , \quad
            \forall n \geq n_\epsilon 
        $$
        Siccome le $f_n$ sono continue, sono tutte integrabili. Inoltre, siccomeme $f_n \rightrightarrows f$, per il teorema di continuit\'{a} del limite, $f$ \`{e} continua e quindi integrabile
        \begin{gather}
            \left| \int_a^b f_n(x)dx - \int_a^b f(x)dx \right| \leq \int_a^b \left| f_n(x) - f(x) \right| dx \\
            \leq \int_a^b \sup_{x\in [a,b]} \mid f_n(x) - f(x) \mid dx \\
            \leq  \; \mid b - a \mid \sup_{x\in [a,b]} \mid f_n(x) - f(x) \mid 
            \quad \rightarrow 0
        \end{gather}
    }
}

\ltheorem{Passaggio al limite sotto il segno di derivata:
    $$
        f'(x) = \lim_{n \rightarrow \infty} f'_n(x)
    $$
    \ldim{
        Manca!
    }
}

\ldef{
    Una successione ${x_n}$ in uno spazio metrico $(X,d)$ prende il nome di \b{successione di Cauchy} se esiste un $N$ tale che:
        $$
        d(x_m, x_n) < \epsilon \quad \forall m,n \geq N, \forall \epsilon > 0
        $$
    In sostanza significa che al tendere all'infinito, lo spazio tra due elementi della successione tende ad annullarsi.
}

\ldef{
    Uno \b{spazio metrico completo} è uno spazio in cui tutte le successioni di Cauchy sono convergenti ad un elemento dello spazio. 
    Viene anche chiamato \b{spazio di Banach}.
    \loss{
        Lo spazio metrico $\mathbb{Q}$ dei razionali con la metrica standard non è completo. Infatti, se prendo la successione i troncamenti di $\sqrt{2}$ definita come $x_n = \frac{\lfloor 10^n \sqrt{2}\rfloor}{10^n}$, è una successione di Cauchy $(1, 1.4, 1.41, \dots)$ che converge a $\sqrt{2}$, un numero non razionale.\\
        Invece, un qualsiasi sottoinsieme chiuso di $\mathbb{R}^n$ è completo.
    }
    \loss{
        $\mathbb{R}^n$ \'{e} completo con la norma euclidea. Siccome poi in $\mathbb{R}^n$ tutte le norme sono equivalenti, qualunque spazio normato in $\mathbb{R}^n$ \'{e} completo. \\
        Segue anche che tutti gli spazi metrici in $\mathbb{R}^n$ in cui la distanza proviene da una norma sono completi.
    }
}

\ldef{
    Una \b{funzione lipschitziana} è una funzione di variabile reale caratterizzata da \emph{crescita limitata}, nel senso che il rapporto tra variazione di ordinata e variazione di ascissa non può mai superare un valore fissato definito come costante di Lipshitz.
}

\ldef{
    Si definisce \b{contrazione} una funzione $f : X \rightarrow X$ tale che esiste $L$ che soddisfa:
    $$
    d(f(x), f(y)) \leq Ld(x,y), \quad L < 1
    $$
    In altre parole, $f$ è una contrazione se \emph{contrae} la distanza tra due elementi $x$ e $y$.
    \loss{
        Ogni contrazione è lipschitziana, e quindi anche continua!
    }
}

\ltheorem{
    Il \b{teorema di Banach-Cacciopolli} dice che dato uno spazio metrico completo non vuoto $(X, d)$, e una sua contrazione $f$, allora la mappa $f$ ammette uno e un solo punto fisso $x^* \in X \mid x^* = f(x^*) $
    \ldim{
        1) Dimostriamo prima l'esistenza, definendo la successione ricorrente:
        $$ x_1 = f(x_0), \quad x_2 = f(x_1), \dots, x_n = f(x_{n-1}) $$
        Usiamo la contrazione per valutare la distanza tra due elementi successivi:
        \begin{gather}
        \begin{split}
        d(x_n, x_{n+1}) = d(f(x_{n-1}), f(x_n)) \leq Ld(x_{n-1}, x_n) = \\
        Ld(f(x_{n-2}), f(x_{n-1})) \leq L^2 d(x_{n-2},x_{n-1}) \leq \dots \leq L^n d(x_0, x_1) 
        \end{split}
        \end{gather}
        Prendiamo due numeri $m, n \in \mathbb{N}, m < n$, e con la disugualianza triangolare:
        \begin{gather}
        \begin{split}
        d(x_n, x_m) \leq d(x_n, x_{n-1}) + d(x_{n-1}, x_m) \leq \sum_{i=m}^{n-1} d(x_i, x_{i+1}) \leq \\
        \leq d(x_0, x_1) \sum_{i=m}^{n-1} = d(x_0, x_1) \sum_{i=0}^{n-m-1} L^{i+m} = L^m d(x_0, x_1) \sum_{i=0}^{n-m-1} L^i
        \end{split}
        \end{gather}
        Siccome $0 < L < 1$, la serie geometrica converge:
        $$ d(x_n, x_m) \leq d(x_0, x_1) \frac{L^m}{1 - L} \rightarrow 0 \quad \text{per} \quad m \rightarrow 0 $$
        che soddisfa il criterio di Cauchy per le successioni. Dato che per ipotesi $X$ \`{e} completo, 
        sappiamo che la successione converge. Siccome $f$ \`{e} un'applicazione continua:
        $$ x^* = \lim_{n \rightarrow \infty} x_n \quad \implies \quad f(x^*) = \lim_{n \rightarrow \infty} f(x_n) = \lim_{n \rightarrow \infty} x_{n+1} = x^* $$
        Percio' abbiamo dimostrato che $f(x^*) = x^*$. \\
        2) Passiamo ora all'unicita', che dimostriamo per assurdo dicendo che esiste un secondo punto $f(y^*) = y^*$:
        $$ d(x^*, y^*) \leq d(f(x^*), f(y^*)) \leq Ld(x^*, y^*) \quad   L \geq 1 $$
        che contraddice l'ipotesi della contrazione $L < 1$.
    }
}

\ldef{
    La \b{serie} di funzioni $\displaystyle \sum_{k=1}^{+\infty} f_k$ non \'{e} altro che la successione $\{s_n\}_k$ delle sue somme parziali.
}

\ldef{
    La \b{convergenza puntuale per le serie} di funzioni si verifica se $\forall x \in I, \forall \epsilon > 0, \exists n_{\epsilon, x} \in \mathbb{N}$ tale che 
        $$ \left| \sum_{k=n+1}^{+\infty} f_k(x) \right| < \epsilon, \quad \forall n > n_\epsilon $$
}

\ldef{
    La \b{convergenza uniforme delle serie} di funzioni si verifica se 
    $ \forall \epsilon > 0, \quad \exists n_\epsilon \in \mathbb{N}$ tale che
    $$ \sup_{x\in I} \left| \sum_{k=n+1}^{+\infty} f_k(x) \right| < \epsilon $$

    \loss{
        La convergenza uniforme implica quella puntuale.
    }
    \loss{
        Se voglio dimostrare che una serie non converge, basta che trovo 
        un $n$ per cui il sup non é 0:
        \[
            \sup_{x\in I} \mid f_{n+1}(x) \mid \nrightarrow 0
        \]
    }
    \loss{
        Se ho una serie della forma 
        \[
            \sum_{k=0}^\infty (-1)^k f_k(x), \quad \text{con} \quad
            f_k(x) \geq 0, f_{k+1}(x) \geq f_k(x), f_k(x) \rightarrow 0
        \]
        allora converge puntualmente $\forall x$ per Leibnitz.\\
        Tuttavia, se ho che $f_k(x) \rightrightarrows 0$, allora converge
        anche uniformemente, poiché  
        \[ 
            \displaystyle \left| \sum_{k=n+1}  (-1)^k f_k(x) \right| \leq \sup \mid f_{n+1} (x) \mid \rightarrow 0
        \]
        
    }
}

\ldef{
    La serie $\displaystyle \sum_{k=1}^{+\infty} f_k(x)$  \b{converge assolutamente} in $I$ se converge (puntualmente) in I la serie $\displaystyle \sum_{k=1}^{+\infty} \mid f_k(x) \mid < + \infty $

    \loss{
        La convergenza assoluta implica quella puntuale. \\ Questo \'{e} verificabile poich\`{e} per il teorema del confronto di serie, vale che 
            $ -\mid f_k(x) \mid \leq f_k(x) \leq \mid f_k(x) \mid $
    }

    \loss{
        Se $ f_k \geq 0$, allora la convergenza puntuale \'{e} uguale a quella assoluta.
    }

}

\ldef{
    La serie $f_k$ si dice \b{totalmente convergente} in I se $\forall k, \exists M_k \geq 0 $ tale che 
        \[ \sum_{k=1}^{+\infty} M_k < \infty \quad \text{e} \quad \mid f_k(x) \mid \leq M_k, \quad \forall x \in I \]

    \loss{
        La serie \'{e} totalmente convergente se e solo se posso prendere $\displaystyle M_k = \sup_{x\in I} \mid f_k(x) \mid $, cosa che poi mi \'{e} molto utile fare quasi sempre.
    }
}

\ldef{
    Il \b{criterio di Cauchy per le serie} dice che la successione $\{s_n\}_n$ converge se e solo se \'{e} di Cauchy.
}

\lprop{
    Se una serie converge totalmente, allora converge anche uniformemente.
    \ldim{
        Sia $ M_k \geq 0$ tale che $ M_k < + \infty$ e $\mid f_k(x) \mid \leq M_k \forall x \in I$. \\
        Uso il criterio di Cauchy uniforme:
        \begin{gather}
        \begin{split}
            \left| \sum_{k=n+1}^{+\infty} f_k(x) \right| \quad \leq \quad \mid f_{n+1}(x) \mid + \dots + \mid f_{n+p}(x) \mid \quad \leq \\
            \leq M_{n+1} + \dots + M_{n+p} = \sum_{k=n+1}^{n+p} M_k
        \end{split}
        \end{gather}
        Ma dato che quest'ultima serie converge in $\mathbb{R}$ uso cauchy per serie numeriche: $\forall \epsilon > 0,  \exists n_\epsilon$ tale che $\displaystyle \sum_{k=n+1}^{n+p} M_k < \epsilon, \quad \forall n > n_\epsilon, \quad \forall p \in \mathbb{N}$\\
        E quindi $\displaystyle \sum_{k=n+1}^{n+p} M_k < \epsilon, \quad  \forall \epsilon > 0, \quad \exists n_\epsilon \text{t.c.} \quad \forall x \in I, \; \forall n > n_\epsilon, \; \forall p \in \mathbb{N}$
    }
}

\ltheorem{
    Il \b{teorema della continuita' del limite per le serie di funzioni} dice che la somma di una serie di funzioni continue ( cio\`{e} $f_k$ continua $\forall k$) che converge uniformemente \'{e} una funzione continua. Questa somma \'{e} $ s(x) = \sum_{k=1}^{+\infty} f_k(x) $
}

\ltheorem{
    Il \b{teorema di integrazione per serie} dice che se
    $ f_k [a,b] \rightarrow \mathbb{R} $ continue, e se $ s_n(x) \rightrightarrows s(x) \text{ in } \left[a,b\right] $, allora:
    \[ \int_a^b s(x)dx = \int_a^b \sum_{k=1}^{+\infty} f_k(x) dx = \sum_{k=1}^{+\infty} \int_a^b f_k(x) dx \]
}

\ltheorem{
    Il \b{teorema di derivazione per serie} dice che data $f_k : I \rightarrow \mathbb{R}$, con
    $ f_k \in C^1(I)$, e dato $S_n(x) = \sum_{k=1}^n f_k(x)$, se $S'_n = \sum_{k=1}^n f'_k(x)_k$
    converge uniformemente, e $ \exists x_0 \in I $ tale che $ S_n(x_0) $ converge (in 
    $\mathbb{R}$), allora: 
    \[
    S_n(x) \rightrightarrows \sum_{k=1}^\infty f'_k(x)  \text{, \quad e \quad }
    \left( \sum_{k=1}^\infty f_k(x) \right)' = \sum_{k=1}^\infty f'_k(x)
    \]
}

\ldef{
    Si dice \b{serie di potenze} una serie di funzioni di questo genere:
    \[
        \sum_{k=0}^\infty a^k (x - x_0)^k
    \]
    Assumiamo $x_0 = 0$, altrimenti basta fissare $y = (x - x_0)$
    \loss{
        Una serie di potenze converge sempre in $ x = 0 $.
    }
    \loss{
        Se una serie di potenze converge in $ \xi \in \mathbb{R} $, allora converge (assolutamente) in $ |x| < |\xi|$.\\
        Analogamente, se \emph{non} converge in $ \xi'  \in \mathbb{R} $, allora non converge in $ |x| > |\xi'|$.
    }
    L'insieme dei valori dove la serie converge prende il nome di \b{insieme di convergenza}. 
    \loss{
        L'insieme di convergenza pu\'{o} essere solo delle seguenti forme: 
        $ \{0\}, (-\rho, \rho), [-\rho, \rho), \left[-\rho, \rho\right], (-\rho, \rho], \mathbb{R}$, dove $\rho$ \'{e} il raggio di convergenza
    }
    \loss{
        La definizione formale del raggio di convergenza \'{e} questa:
        $ \rho = \sup \{ |x| \mid x \in A \}$, dove $A$ \'{e} l'insieme di convergenza.
    }
}

\ldef{
    Il \b{criterio della radice} dice che il raggio di convergenza $ \rho \geq 0 $ di una serie 
    di potenze \'{e} uguale a $\frac{1}{l}$ dove 
    \[
    l = \limsup_{k\to\infty} \sqrt[k]{\vphantom{\sum} \left| a_k \right| }
    \]
    \loss{
        Il limsup \'{e} il limite maggiore di tutte le possibili sottosuccessioni. Per il teorema di Bolzano-Weierstrass, esiste sempre almeno una sottosuccessione convergente, e quindi esiste sempre il limsup.
        }
}

\ldef{
    Il \b{criterio del rapporto} dice che data una serie di potenze, se esiste
    \[
        l = \lim_{k\to+\infty} \frac{|a_{k+1}|}{|a_k|}
    \]
    allora il raggio di convergenza \'{e} $ \rho = \frac{1}{l}$
}

\ltheorem{
    Il \b{teorema di Abel} dice che se una serie numerica $ \sum_k^\infty a_k \rho^k $ con $ \rho > 0 $ converge, allora la serie di potenze $ \sum_k^\infty a_k x^k$ converge uniformemente in $[-\rho + \delta, \rho ], \; \forall \delta > 0$. \\
    Se invece $ \rho < 0$, allora la serie converge uniformemente in $[-\rho, \rho - \delta], \; \forall \delta > 0$.
}

\ldef{
    Data una serie di potenze, si dice \b{serie derivata } la serie:
    $$
    \sum_k^\infty k a_k x^{k-1}
    $$
}

\ltheorem{
    Il \b{raggio di una serie e della sua derivata} \'{e} lo stesso.
    \ldim{
        Consideriamo $\sum k a_k k^k = x \sum k a_k x^{k-1}$.
        Il raggio di convergenza di queste due serie \'{e} lo stesso poich\'{e} la parte indipendente da $x$ \'{e} la stessa. Confrontiamo $\sum k a_k x^k$ con $\sum a_k x^k$, usando il criterio della radice. Anche qui i due raggi di convergenza sono uguali poich\'{e} $\limsup_{k\to\infty} \sqrt[k]{|a_k|} = \limsup_{k\to\infty} \sqrt[k]{k|a_k|}$
    }
}

\ltheorem{
    Se una serie ha raggio di convergenza $ \rho > 0$, allora sia la derivata che l'integrale della somma della serie hanno lo stesso raggio di convergenza $\rho$.
}


\ldef{
    Una funzione $f : I \rightarrow \mathbb{R}; \quad f \in C^\infty (I) $, si dice \b{sviluppabile in serie di Taylor} se \'{e} possibile scriverla nella forma 
    $$ f(x) = \sum_{k=0}^\infty a_k ( x - x_0 )^k $$
    con $x_0$ fissato e $ x \in ( x_0 - \rho, x_0 + \rho )$, per $\rho > 0$.\\
    Se in particolare $x_0 = 0$, allora prende il nome di \b{serie di MacLaurin}.
    \loss{
        Calcolando le derivate in $x_0$ otteniamo i termini $a_k$:
        \[
            a_k = \frac{f^{(k)} (x_0)}{k!}, \quad \forall k > 0
        \]
    }
    \loss{
        Esempi di funzioni sviluppabili in Taylor:
        \begin{align*}
            log(1+x)=&\sum \frac{(-1)^k x^{k+1}}{k+1} &\frac{1}{1+x}& = \sum x^k, \quad x \in (-1, 1)\\
            \frac{1}{1+x^2} = &\sum x^{2k} &arctan(x)&= \sum \frac { (-1)^k x^{2k + 1}}{ 2k+1}\\
            \frac{-1}{(1 - x)^{-2}} = &\sum kx^{k-1} & e^x &= \sum \frac{x^n}{n!}\\
            sin(x) =&\sum\frac{(-1)^{n-1}x^{2n -1}}{(2n -1)!} & cos(x) &= \sum\frac{(-1)^{n}2^n}{(2n)!}
        \end{align*}
        Notare che l'ultima \'{e} stata ottenuta integrando la terza.\\
        In linea di massima, ognuna di queste pu\'{o} essere derivata/integrata a piacere.
    }
}

\ltheorem{
    Il \b{teorema di sviluppabilit\'{a} in serie di Taylor} dice che se $f$ \'{e} dotata delle derivate di ogni ordine e se $\exists M, L > 0$ tali che 
    \[
    \left| f^{(k)} (x) \right| \leq M\cdot L^k, \quad \forall k = 0,1,2 \dots, \quad \forall x \in (a,b)
    \]
    allora $f$ \'{e} sviluppabile in $x_0$ per ogni $x_0 \in (a, b)$, per $x \in (a,b)$.
    \ldim{
        Vogliamo dimostrare che il resto dello sviluppo di Taylor tende a 0:
        \[
        R(n) = f(x) - \sum_{k=0}^n \frac{f^{(k)} (x_0)}{k!} (x - x_0)^k \quad \longrightarrow 0, \quad n \rightarrow \infty
        \]
        Ora scriviamolo in forma di Lagrange:
        \[
        R_n(x) = \frac{\left| f^{(n+1)} (\xi) \right| } { (n + 1)!} (x - x_0)^{n + 1}, \quad \xi \in (x, x_0)
        \]
        Siccome il valore massimo di $(x - x_0)$ \'{e} in $(b - a)$:
        \[
        \left| R_n(x) \right| \leq \frac{ \left| f^{(n + 1)} (\xi) \right|} { (n + 1) !} ( b - a)^{n+1} 
        \leq \frac{ML^{n+1}}{(n+1)!} (b-a)^{n+1} \longrightarrow 0
        \]
} 
}

\ldef{
    Una \b{curva} \'{e} un'applicazione continua $\varphi : I \rightarrow \mathbb{R}^d, d\in \mathbb{N}$. L'immagine della curva, anche detto \b{sostegno}, \'{e} l'insieme dei punti per cui passa la curva, definito come $\{\varphi(t) \in \mathbb{R} | t \in I \}$.
    \loss{
        $\varphi$ \'{e} continua se $t \rightarrow \varphi_i(t) \text{ \'{e} continua } \forall i = 1,2,3,\dots$
    }
    \loss{
        A uno stesso sostegno possono appartenere varie curve.
    }
}

\ldef{
    Una curva di dice \b{semplice} se non si auto interseca, cio\'{e} se $\varphi : \mathring{I} \longrightarrow \mathbb{R}^d$ \'{e} iniettiva, dove $\mathring{I}$ \'{e} $I$ senza estremi.
}

\ldef{
    Una curva è derivabile se ogni componente è derivabile. Il vettore $\varphi' (t) = (\varphi_1'(t), \varphi_2'(t), \dots, \varphi_d'(t))$ è detto \b{vettore velocit\'{a}.}
}

\ldef{
    Una curva si dice \b{regolare} se $ \varphi'(t) \neq 0 \quad \forall t \in \mathring{I} $
}

\ldef{
    Il versore $\displaystyle T(t) = \frac{\varphi'(t)}{ | \varphi'(t)|}$ è detto \b{tangente}.
}

\ldef{
    La \b{lunghezza di una curva} è definita nel seguente modo:
    \[
        L(\varphi) = \sup 
        \begin{cases}
            L(\pi) \mid \pi \text{ è una poligonale inscritta} \\
            \text{con punti } t_0< t_1< \dots < t_k < \infty 
        \end{cases}
    \]
    Dove la poligonale è una curva fatta di segmenti, e una poligonale inscritta è una poligonale che passa per tutti i punti di una data curva (con segmenti infinitesimali)

}

\ltheorem{
    Se $\varphi \colon [a,b] \longrightarrow \mathbb{R}^d$ è una curva regolare, allora:
    \[
        L(\varphi) = \int_a^b \mid \varphi'(t) \mid dt < +\infty
    \]
    \loss{
        Il teorema vale anche se la curva è regolare solo a tratti.
        In questo caso dovr\'{o} spezzare l'integrale nei vari tratti.
    }
    \loss{
        La lunghezza della curva non è la lunghezza del sostegno.
        Infatti, una curva potrebbe fare vari giri.
    }
}


\ldef {
    Le curve \b{cartesiane} hanno il sostegno che è una funzione 
    $f \colon  \mathbb{R} \rightarrow \mathbb{R}$
    Quindi sono curve definite in questo modo: 
    $\varnothing(t) = (t, f(t))$.
    \loss{
        Tutte le curve cartesiane sono semplici, in quanto essendo $f$ una funzione,
        non pu\'{o} avere due risultati diversi sulla stessa $x$.
    }
    \loss{
        Se $f \in C^1$, allora $\varnothing$ è regolare: $\varnothing' (t) = (1, f'(t)) \neq 0$.
    }
    \loss{
        La lunghezza delle curve cartesiane si calcola come:
        \[
            L(\varnothing) = \int_a^b \sqrt{1 + (f'(t))^2} dt
        \]
    }
}

\ldef{
    Le curve \b{polari} sono quelle curve che si possono esprimere come 
    funzione dell'angolo con l'origine $\rho(\theta) \colon \rightarrow (0, +\infty)$.
    Quindi la curva è definita come:
    $\varnothing(\theta) = (\rho(\theta)\cos\theta, \rho(\theta)\sin\theta)$
    \loss{
        Tutte le curve polari sono regolari, in quanto l'angolo cambia sempre.
    }
    \loss{
        La lunghezza delle curve polari si calcola come:
        \[
            L(\varnothing) = \int_a^b \sqrt{\rho(\theta)^2 + \rho'(\theta)^2} d\theta
        \]
    }
 
}

\ldef{
    Si definiscono le seguenti \b{funzioni iperboliche}:
    \begin{align*}
        \cosh &= \frac{e^t + e^{-t}}{2},&& \text{pari},& 
        \text{Immagine:}\quad& [1, +\infty]
       \\
        \sinh &= \frac{e^t - e^{-t}}{2},&& \text{dispari},& 
        \text{Immagine:}\quad& [-\infty,  +\infty]
    \end{align*}
    Si chiamano iperbolici perch\'{e} se applico la seguente mappatura:
    \begin{align*}
        \cos &\rightarrow \cosh
        &\sin &\rightarrow \sinh
        \\
        \cos^2 &\rightarrow \cosh^2
        &\sin^2 &\rightarrow -\sinh^2
    \end{align*}
    Tutte le identit\'{a} trigonometriche sono ancora verificate.
    \loss{
        \[
            \cosh' = \sinh \quad \quad \sinh' = \cosh
        \]
    }
    \vspace{-1em}
    \loss{
        \begin{gather*}
            arccosh(x) = log(x + \sqrt{x^2 - 1})
            \\
            arcsinh(x) = log(x + \sqrt{x^2 + 1})
        \end{gather*}
    }
}

\ldef{
    Due curve 
    $\varnothing \colon [a, b] \rightarrow \mathbb{R}^d $ e 
    $\varphi \colon [c, d] \rightarrow \mathbb{R}^d $
    si dicono \b{equivalenti} se esiste una funzione 
    $h \colon [c, d] \rightarrow [a, b]$
    continua e univoca tale che 
    $\varnothing(h(t)) = \varphi(t), \quad \forall t \in [c,d]$.
}

\ltheorem{
    Due curve equivalenti hanno la stessa lunghezza.
}


\ldef{
    $\v{P}$ é \b{punto interno} di $E$ se
    $\exists r > 0 $ t.c. $ B(\v{P}, r) \subseteq E$.\\
    $\v{P}$ é \b{punto esterno} di $E$ se $\v{P}$ é interno a $E^c$.\\
    $\v{P}$ é \b{punto di frontiera} se 
    $\forall r > 0, B(\v{P})$ contiene sia punti di E che del complementare.\\
    $\v{P}$ é \b{punto di accumulazione} per $E$ se 
    $\forall r > 0, (B(\v{P}, r) \backslash \{\v{P}\} \; \cap E \neq \emptyset  $.
    \\
    Ogni punto di $E$ che non é di accumulazione si dice \b{punto isolato}.
}

\ldef{
    Un insieme $E$ si dice \b{aperto} se ogni suo punto é interno.\\
    Un insieme $E$ si dice \b{chiuso} se il suo complementare é aperto.\\
    Un insieme $E$ é \b{limitato} se $\exists r > 0$ t.c. 
    $E \subseteq B(\v{0}, r)$.\\
    La \b{chiusura} di un insieme $E$ é il piú piccolo insieme chiuso 
    $E'$ tale che $E \subseteq E'$.
    \loss{
        L'unione e l'intersezione di due insiemi aperti é un insieme aperto.
        L'unione e l'intersezione di due insiemi chiusi é un insieme chiuso.
    }
}





\end{multicols}
\msection{Funzioni a piú variabili}
\begin{multicols}{2} 




\ldef{
    Tutte le funzioni del tipo 
    $f \colon \mathbb{R}^n \longrightarrow \mathbb{R}^d$
    si chiamano \b{vettoriali} se $n > 1$, e \b{a piú variabili} se $d > 1$.
}

\ldef{
    Sia $\v{x_0} \in A \subseteq \mathbb{R}^n$, con $\v{x_0}$ punto di 
    accumulazione di $A$.
    Sia $f\colon \mathbb{R}^n \longrightarrow \mathbb{R}^d$, e sia
    $\v{L} = (L_1, \ldots, L_d) \in \mathbb{R}^d$ .
    Definiamo:
    \begin{gather*}
        \lim_{\v{x}\rightarrow\v{x_0}} f(\v{x}) = \v{L} \quad \text{se}\\
        \forall \epsilon > 0, \exists \delta > 0 \text{ tale che se } 
        | \v{x} - \v{x_0} | \; < \delta \text{ allora } 
        | f(\v{x}) - \v{L} | < \epsilon
    \end{gather*}
    \loss{
        $f(\v{x}) \rightarrow \v{L}$ se e solo se $f_i(x) \rightarrow L_i, 
        \quad \forall i = 1,\dots,d$\\
        Manca la dimostrazione!
    }
    \loss{
        Per dimostrare che non esiste il limite devo trovare due successioni 
        $\v{P_n} \rightarrow c $ e $\v{Q_n} \rightarrow c $ tali che 
        $f(\v{P_n}) \rightarrow \v{l} $ e $ f(\v{Q_n}) \rightarrow \v{l'}$.
        Se $\v{l} \neq \v{l'}$, allora il limite non esiste.
    }
    \loss{
        Se bisogna calcolare un limite per $(x,y) \rightarrow (x_0, y_0)$ 
        invece che $(x, y) \rightarrow (0,0)$, basta che impongo
        $x' = x - x_0, y' = y - y_0$, e scrivo la funzione $f(x,y) = f'(x',y')$.
        Poi calcolo il limite di $f'$ per $(x', y') \rightarrow (0,0)$.
    }
    \loss{
        Per calcolare il limite di una funzione a piú variabili spesso
        aiuta molto imporre $y = x^\beta$, con il giusto $\beta$.
    }
    \loss{
        Quando si deve calcolare anche il limite in base a un dato valore
        $\alpha$, attenzione a non fare maggiorazioni improprie perché
        si potrebbero perdere alcuni valori di $\alpha$.
    }
}

\ldef{
    In $\mathbb{R}^2$, si definisce il limite a infinito:
    \begin{gather*}
        \lim_{|(x,y)| \rightarrow \infty} f(x,y) = l \quad \text{se}\\
        \forall \epsilon > 0, \exists R > 0 \text{ tale che se } 
        | (x,y) | \geq R \text{ allora } 
        | f(x,y) - l | < \epsilon
    \end{gather*}
}

\ldef{
    In $\mathbb{R}^2$, se il limite a un punto $(x_0, y_0)$ tende a $\pm\infty$:
    \begin{gather*}
        \lim_{(x,y) \rightarrow (x_0, y_0)} f(x,y) = \pm\infty\quad\text{se}\\
        \forall M > 0, \exists \delta > 0 \text{ tale che se } 
        | (x,y) - (x_0, y_0) | < \delta \text{ allora } \\
        f(x,y) - l > M \text{ (oppure $< -M$ nel caso di $-\infty$)}
    \end{gather*}
}

\ldef{
    In $\mathbb{R}^2$, se il limite a $\pm\infty$ tende a $\pm\infty$:
    \begin{gather*}
        \lim_{(x,y) \rightarrow (x_0, y_0)} f(x,y) = \pm\infty\quad\text{se}\\
        \forall M > 0, \exists N > 0 \text{ tale che se } 
        | (x,y) - (x_0, y_0) | > N \text{ allora } \\
        f(x,y) - l > M \text{ (oppure $< -M$ nel caso di $-\infty$)}
    \end{gather*}
}


\ldef{
    Un punto $(x, y)$ puó essere espresso anche in base a un altro $(x_0, y_0)$
    attraverso le coordinate polari. In questo caso, si ottiene::
    \begin{align*}
        x &= x_0 + \rho \cos\theta \\
        y &= y_0 + \rho \sin\theta
    \end{align*}
    Quindi $(x,y) \rightarrow (x_0, y_0) \equiv | (x - x_0, y - y_0) | 
    \rightarrow 0 \equiv \rho \rightarrow 0$ .
    \loss{
        Attenzione! Non si fissare $\theta$ e poi ottenere il limite, poiché il 
        limite deve essere costante per ogni $\theta$! \\
        Se il limite non é costante,  é molto probabile che non esiste.
    }
}

\ldef {
    Sia $f:\in \mathbb{R}^n \rightarrow \mathbb{R}^d$ e $\v{c}$ punto di accumulazione in $A$ (con 
    $\v{c} \in A$. Si dice che $f$ é \b{continua} in $A$ se:
    \[
        \forall \epsilon > 0, \exists \delta > 0 \text{ tale che se } | \v{x} - \v{c} | < \delta 
        \text{ allora } |f(\v{x}) - f(\v{c}) |_{\mathbb{R}^d} < \epsilon
    \]
    \loss {
        Come per i limiti, vale che $f : A \in \mathbb{R}^n \rightarrow \mathbb{R}^d$
        é continua in $\v{c} \in A$ se e solo se $f_i : \mathbb{R} \rightarrow \mathbb{R} $ 
        é continua $ \forall i = 1, 2, \dots d$ , cioé se sono continue tutte le sue 
        componenti.
    }
    \loss {
        Data $f : A \in \mathbb{R}^n \rightarrow \mathbb{R}^d$, con $A$ aperto \slash chiuso, 
        si ha che se $f$ é continua, allora anche $f^{-1}$ é aperto \slash chiuso, rispettivamente.
    }
}

\ldef{
    Data $ f :  A \in \mathbb{R}^n \rightarrow \mathbb{R}$, con $ \v{x}_0 $ punto interno, 
    e dato $\v{v} \in \mathbb{R}^n$, con $ |v| = 1$, allora la \b{derivata direzionale} di $f$ in 
    $x_0$ verso $\v{v}$ é il seguente limite (se esiste):
    \[
        \lim_{t \rightarrow 0} \frac{ f(\v{x}_0 + t\v{v}) - f(\v{x}_0)  }{t} \equiv D_\v{v} f(x_0) \equiv 
        \frac{df}{d\v{v}} (\v{x}_0) \equiv d_\v{v} f (\v{x}_0) 
    \]
    \loss{
        Potrebbe essere comodo scrivere $\v{v} = (\cos \theta , \sin\theta )$.
    }
    \loss {
        Ponendo $\varphi(t) = f(\v{x}_0 + t\v{v})$, si ottiene $ D_\v{v} f(x_0) = \varphi'(0)$.
    }
}

\ldef{
    Prende il nome di \b{derivata parziale} di $f$ in $x_0$ rispetto alla variabile $ x_i$ la 
    derivata direzionale usando $ \v{v} = (0, 0, \dots, 1, \dots, 0, 0)$, con l'1 all'i-esima 
    posizione.
}

\ldef{
    Se nel punto $x_0$ esistono tutte le derivate parziali (quindi $f$ é derivabile lungo tutti 
    gli assi), allora si dice che $f$ é \b{derivabile} in $x_0$.
    Se risulta derivabile per $\forall x_0 \in A$, allora si dice derivabile in $A$.
    \loss{
        Per verificare l'esistenza delle derivate parziali, bisogna usare
        il limite del rapporto incrementale.
    }
}

\ldef{
    Il \b{gradiente} é il vettore formato dalle derivate parziali:
    \[
        D f(x_0) = \triangledown f(x_0) = (
            \frac{df}{d\v{x_0}} f(x_0),
            \frac{df}{d\v{x_1}} f(x_0),
            \dots,
            \frac{df}{d\v{x_n}} f(x_0))
    \]
}

\ldef{
    Data $ f : \mathbb{R}^n \rightarrow \mathbb{R} $ si dice $f \in C^1$ in $x_0$ se é derivabile 
    e $ \triangledown f $ é continuo in $x_0$ .
    \loss{
        Ricordarsi che il gradiente é continuo se e solo se tutte le derivate parziali sono continue!
    }
}

\ldef{
    Il piano tangente a una superficie si trova con:
    \[
        z = f(x_0, y_0) + \frac{df}{dx} (x_0, y_0)(x - x_0) + \frac{df}{dy} (x_0, y_0)(y - y_0)
    \]
    \loss{
        In sostanza mi fermo al primo passo di approssimazione dello sviluppo di Taylor, dove 
        approssimo una superficie con un piano
    }
}

\ldef{
    Data $ f : A \in \mathbb{R}^n \rightarrow \mathbb{R}$, con $A$ aperto, con $x_0 \in A$, $f$ si 
    dice \b{differenziabile} in $x_0$ se esiste $\v{a} \in \mathbb{R}^n $ tale che:
    \[
        f(\v{x}_0 + \v{h}) - f(\v{x}_0) = \v{a} \cdot \v{h} + o(|\v{h}|), 
        \text{ dove } \v{h} \in \mathbb{R}^n \text{ e } \v{h} \rightarrow 0
    \]
    Inoltre, $f$ si dice differenziabile in $A$ se é differenziabile in $\forall x_0 \in A$.
    \loss{
        Per verificare la differenziabilit\'{a} di una funzione in un punto $\v{x}_0$, verificare che il seguente limite valga 0:
        \[
            \lim_{\v{h} \rightarrow 0} \frac{f(\v{x}_0 + \v{h}) - f(\v{x}_0) - \triangledown f(\v{x}_0) \cdot \v{h} }{|\v{h}|}
        \]
    }
}

\ldef{
    Il \b{differenziale} di $f$ in $x_0$ é l'applicazione lineare 
    $d f(\v{x}_0) : \mathbb{R}^n \rightarrow \mathbb{R} \quad 
    \quad \v{h} \rightarrow \v{a} \cdot \v{h}$
}

\ltheorem{
    Data $ f:A\in\mathbb{R}^n\rightarrow\mathbb{R}$, con $A$ aperto, con $\v{x}_0\in A$, se $f$ é differenziabile in $\v{x}_0$, allora vale che:

    - $f$ é continua in $\v{x}_0$
    \ldim{
        $ f(\v{x}_0 + \v{h}) - f(\v{x}_0) = \v{a}\cdot \v{h} + o(|\v{h}|) \rightarrow 0$
        $ \implies \lim_{t \rightarrow 0} f(\v{x}_0 + \v{h}) = f(\v{x}_0)$
    }

    - $f$ é derivabile direzionalmente in $\v{x}_0$, in particolare é derivabile e 
    $\v{a} = \triangledown f(\v{x_0})$.
    \ldim{
        Pongo $\v{h} = t\v{e}_j$, $t \in \mathbb{R} $, $t \rightarrow 0$, $ \v{e}_j = (0, 0, \dots , 1, \dots )$ con un 1 alla j-esima posizione.
        \begin{gather*}
            f(\v{x}_0 + t\v{v}) - f(\v{x}_0) = \v{a} \cdot t \v{e}_j + o(|t|) \implies \\
            \frac{1}{t} (f(\v{x}_0 + t\v{e}_j) - f(\v{x}_0)) = \v{a}\,\v{e}_j + \frac{o(|t|)}{t}\implies \\
            \lim_{t \rightarrow 0} f(\v{x}_0 + t\v{e}_j) - f(\v{x}_0) = \v{a}_j
        \end{gather*}
        Tuttavia, questo dimostra solo la derivabilitá.
        Per dimostrare che é derivabile direzionalmente, vedere la prossima dimostrazione.
        
    }

    - $ D_\v{v} f(\v{x}_0) = \triangledown f(\v{x}_0) \cdot \v{v} $
    \ldim{
        Pongo $ \v{h} = t\v{v})$, $ \implies f(x_0 + tv) - f(x_0) = \triangledown f(x_0) \cdot tv + o(|t|)$
        Dividendo entrambi i membri per $t \implies D_\v{v} f(x_0) = \triangledown f(x_0) \cdot v, 
        \forall v \in \mathbb{R}^n $.
        (In pratica ho dimostrato che la derivata direzionale esiste sempre, e che quindi la $f$ é derivabile direzionalmente)
    }
}


\ldef{
    Data $f:A\in\mathbb{R}^n\rightarrow\mathbb{R}$, con $\v{x}_0 \in A$, con $f$ 
    differenziabile in $\v{x}_0$, allora: 
    \[
        D_\v{v} f(\v{x}_0) = \triangledown f(\v{x}_0)) \cdot \v{v} = 
        | \triangledown f(\v{x}_0)) | |\v{v}| \cos\beta = | \triangledown f(\v{x}_0)) | \cos\beta
    \]
    é massimo quando $ \cos\beta = 1$, cioé quando $ \beta = 0$, cioé quando 
    $ \triangledown f(\v{x}_0)) $ é parallelo a $\v{v}$ (e hanno lo stesso verso).
    In questo caso vorrá dire che:
    \[
        \v{v} = \frac{\triangledown f(\v{x}_0))} { |\triangledown f(\v{x}_0))|}
    \]
}

\ldef{
    Se $f$ é differenziabile in $\v{x}_0$, allora l'iperpiano tangernte si ottiene come:
    \[
        x_{n+1} = f(\v{x}_0)) + \triangledown f(\v{x}_0)) (\v{x} - \v{x})
    \]
    \ldim{
        Sapevamo che:
        \[
            f(\v{x}_0) - \v{h}) = f(\v{x}_0) + \triangledown f(\v{x}_0) \cdot \v{h} + o(|h|) 
        \]
        Tuttavia se poniamo $ \v{h} = (\v{x} - \v{x}_0)$:
        \[
            f(\v{x}) = f(\v{x}_0) + \triangledown f(\v{x}_0) (\v{x} - \v{x}_0 ) + o(| \v{x} - \v{x}_0 |)
        \]
    }
}

\ltheorem{
    Il teorema del \b{differenziale totale} dice che data $f:A\in\mathbb{R}^n\rightarrow\mathbb{R}$, 
    con $\v{x}_0 \in A$, se $f \in C^1 $ in $\v{x}_0$, allora $f$ é differenziabile in $\v{x}_0$.
}

\end{multicols}
\msection{Funzioni a piú variabili 2 la vendetta}
\begin{multicols}{2} 

\ltheorem{ 
    Il teorema del \b{differenziale totale} dice che data $f:A\in\mathbb{R}^n\rightarrow\mathbb{R}$, 
    se $f$ é derivabile e $f_{x_1}, f_{x_2}, \dots, f_{x_n}$ sono continue in $\v{x}_0$, allora
    $f$ é differenziabile in $\v{x}_0$.

    \v{Corollario}: se $f \in C^1 (A)$, allora $f$ é differenziabile in $A$.



    \ldim{
        Il teorema verrá dimostrato per $n_2$:
        \begin{gather*}
            f(x_0 + h, y_0 + k) - f(x_0 , y_0) =\\ f(x_0 + h, y_0 + k) - f(x_0, y_0 + k) + f(x_0, y_0 + k) - f(x_0, y_0)
        \end{gather*}
        Adesso poniamo $ g(h) = f(x_0 + h, y_0 + k)$.\\
        Siccome per ipotesi $g$ é derivabile, uso il teorema di Lagrange:
        \[
            g(h) - g(0) = g'(x_h)(h - 0), \text{ con } x_h \in (0, h)
        \]
        Poniamo inoltre $ l(k) = f(x_0, y_0 + k)$\\
        Siccome anche $l$ é derivabile, sempre per Lagrange:
        \[ 
            l(h) - l(0) = l\,'(y_k)(k-0) \text{ con } y_k \in (0, k)
        \]
        Svolgiamo le derivate in entrambi i membri per $l$ e $k$:
        \[
            g'(x_h) = \frac{df}{dx} (x_0 + x_h, y_0 + k), \quad \quad
            l\,'(y_k) = \frac{df}{dy}(x_0, y_0 + y_k)
        \]
        Impostiamo $\varepsilon_h = x_0 + x_h$, e $ \varepsilon_k = y_0 + y_k$:
        \[
            f(x_0 +h, y_0 + k) - f(x_0, y_0) = \frac{df}{dx} (\varepsilon_h, y_0 + k)h +
            \frac{df}{dy} (x_0, \varepsilon_k)k
        \]
        Adesso dobbiamo dimostrare che il seguente limite valga 0:
        \begin{gather*}
            \lim_{(h,k) \rightarrow (0,0)} 
                \frac{
                    f(x_0 +h, y_0 + k) - f(x_0, y_0) - 
                    \frac{df}{dx} (x_0 , y_0 + k)h + \frac{df}{dy} (x_0, y_0 )k
                }{
                    \sqrt{h^2 + k^2}
                } = \\
                \frac{
                    (\frac{df}{dx} (\varepsilon_h, y_0 + k) - \frac{df}{dx}
                    (x_0, y_0 ))h + (\frac{df}{dy} (x_0, \varepsilon_k) -
                    \frac{df}{dy} (x_0, y_0 ))k
                }{
                    \sqrt{h^2 + k^2}
                } \rightarrow 0?
        \end{gather*}
        Siccome dobbiamo verificare che il limite vada a 0, possiamo maggiorare
        il limite con la sommma dei moduli: 
        \begin{align*}
            \leq
            \left| \frac{df}{dx} (\varepsilon_h, y_0 + k) - 
            \frac{df}{dx} (x_0, y_0 ) \right| 
            \frac{
                |h|
            }{
                \sqrt{h^2 + k^2}
            }  \\
            \left| \frac{df}{dy} (x_0, \varepsilon_k) - \frac{df}{dy} 
            (x_0, y_0 ) \right|
            \frac{
                |k|
            }{
                \sqrt{h^2 + k^2}
            } 
        \end{align*}
        Ora abbiamo che $\varepsilon_h\rightarrow 0$, $(y_0+k)\rightarrow 0$,
        $\varepsilon_k \rightarrow 0$, $ \frac{|h|}{\sqrt{h^2 + k^2}}
        \rightarrow 1$,\\
        perció tutto il limite tende a 0.
    }

    \loss{
        Tutte le $f \in C^1$ sono differenziabili. Non é detto il contrario.
    }
    \loss{
        Il teorema dice che se $f$ é differenziabile in $\v{x}_0$, allora 
        \[
            D_v f(\v{x}_0) = \triangledown f(\v{x}_0) \cdot \v{v}, 
            \quad \forall\v{v} 
        \]
        Questo vuol dire che se il gradiente é 0, e una delle derivate
        direzionali é diversa da 0, allora vuol dire che la funzione non é
        differenziabile.
    }
}

\ldef{
    Data $f:A\in\mathbb{R}^n\rightarrow\mathbb{R}^k$, si dice che $f$ é
    derivabile/differenziabile /$C^m$ se lo é componente per componente.
}

\ldef{
    Siano le seguenti funzioni:
    \begin{align*}
        f&:A\subseteq\mathbb{R}^n\rightarrow B\subseteq\mathbb{R}^k && 
        \text{ con } f \text{ differenziabile in } A
        \\
        g&:B\subseteq\mathbb{R}^k\rightarrow \mathbb{R}^d && \text{ con }
        g \text{ differenziabile in } B
    \end{align*}
    Se $f$ é derivabile, allora si definisce \b{jacobiana} la matrice $k \times
    n$  delle derivate parziali:
    \[
        J_f(x_0) =
        \begin{bmatrix}
            \frac{\delta f_1}{\delta x_1} &
            \dots &
            \frac{\delta f_1}{\delta x_n} & \\
            \dots &
            \dots &
            \dots & \\
            \frac{\delta f_k}{\delta x_1} &
            \dots &
            \frac{\delta f_k}{\delta x_n} & \\
        \end{bmatrix}
        =
        \begin{bmatrix}
            \triangledown f_1(x_0) \\
            \dots  \\
            \triangledown f_k(x_0) \ \
        \end{bmatrix}
    \]
    Sia $h:A \subseteq \mathbb{R}^n \rightarrow \mathbb{R}^d$, definita
    come $\v{x} \rightarrow g(f(\v{x}))$.\\
    Allora $h$ é differenziabile in $A$ e inoltre:
    \[
        J_h(\v{x}) = J_g(f(\v{x})) \cdot J_f(\v{x}) =
        \begin{bmatrix}
            \dots &
            \dots &
            \dots & \\
            \dots & 
            \frac{\delta h_i}{\delta x_j} & 
            \dots & \\
            \dots &
            \dots &
            \dots & \\
        \end{bmatrix}
    \]
    A questo punto possiamo calcolarci l'elemento generico:
    \[
        \frac{dh_i}{dx_j} =
        \sum_{s = 1}^k d_sg_i(f(\v{x})) \cdot d_jf_s(\v{x}), 
        \quad \forall i = 1, \dots, d
        \quad \forall j = 1, \dots, n
    \]
}

\ldef{
    Data $g:\mathbb{R}^2 \rightarrow \mathbb{R} $, con $g \in C^1$, 
    si definisce \b{l'insieme di livello} come:
    \[
        \left\{ (x,y) \in \mathbb{R}^2 \mid g(x,y) = c \right\}
        \text{ con } c \in \mathbb{R}
    \]
    \loss{
        Per valori di $c$ "buoni", l'insieme di livello é il sostegno
        di una curva regolare.
    }
}
\lprop{
    Il gradiente di una funzione é ortogonale al suo insieme di livello.

    \ldim{ 
    Lavoriamo prima in $\mathbb{R}^2$, e parametrizziamo la curva
    rappresentante l'insieme di livello:
    \[
        \varnothing : [a, b] \rightarrow \mathbb{R}^2 \quad t \rightarrow
        \begin{cases}
            x(t)\\
            y(t)
        \end{cases}
        \text{ con } Im(\varnothing) = \{ g(x,y) = c \}
    \]
    Il processo é simile a quello di tagliare a "fette" la funzione.\\
    Per costruzione abbiamo che $c = g(x(t), y(t))$.\\
    Se deriviamo rispetto a $t$:
    \[
        \triangledown g(\varnothing(t)) \cdot \varnothing'(t) = 0
    \]
    Che significa che $ \triangledown g$ é ortogonale alla tangente
    in $t$ dell'insieme di livello. (Poiché il prodotto scalare é 0
    solo se i vettori sono ortogonali).\\
    Ora proviamo la stessa identica cosa in $\mathbb{R}^3$.
    Definiamo $g : \mathbb{R}^3 \rightarrow \mathbb{R}$, 
    e il suo insieme di livello $\{ (x,y,z) \mid g(x,y,z) = c \}$.\\
    Questo insieme adesso é rappresentato da una superficie regolare, 
    che parametrizziamo in questo modo:
    \[
        \pi : \mathbb{R}^2 \rightarrow \mathbb{R}^3 \quad
        (s, t) \rightarrow 
        \begin{cases}
            x(s, t)\\
            y(s, t)\\
            z(s, t)
        \end{cases} 
    \]
    Il piano tangente dell'insieme di livello é generato da
    \[
        \left( 
        \frac{\delta x(s,t)}{\delta s}, 
        \frac{\delta y(s,t)}{\delta s}, 
        \frac{\delta z(s,t)}{\delta s}
        \right) \text{ e }
        \left( 
        \frac{\delta x}{\delta t}, 
        \frac{\delta y}{\delta t}, 
        \frac{\delta z}{\delta t}
        \right) 
    \]
    Definiamo $h : \mathbb{R}^2 \rightarrow \mathbb{R}$ come
    \[
        h(s, t):= g(\pi(s, t)); \quad c = h(s,t) = g(\pi(s,t))
    \]
    Facendo la derivata:
    \[
        J_h(s,t) = J_g(\pi(s,t)) \cdot J_\pi(s,t) =
        \triangledown g(\pi(s,t)) \cdot
        \begin{bmatrix}
            \sfrac{\delta x}{\delta s} &
            \sfrac{\delta y}{\delta s} \\
            \sfrac{\delta z}{\delta s} & 
            \sfrac{\delta x}{\delta t} \\
            \sfrac{\delta y}{\delta t} &
            \sfrac{\delta z}{\delta t} 
        \end{bmatrix} = (0, 0)
    \]
    Cioé $ \triangledown g$ é ortogonale ai due vettori che generano
    il piano tangente alla superficie di livello, e quindi
    $ \triangledown g$ é ortogonale alla superficie di livello.
    }
}

\ldef{
    Se esistono tutte le derivate seconde parziali, la funzione si dice
    \b{derivabile due volte}. \\
    La notazione usata é la seguente:
    \[
        \frac{d}{dx_j} ( \frac{df}{dx_i} ) = \frac{d^2}{dx_idx_j} =
        f_{x_i,x_j}
    \]
}

\ldef{
    Si dice matrice \b{Hessiana} la matrice formata da tutte le possibili
    derivate seconde:
    \[
        H_f(\v{x}) = 
        \begin{bmatrix}
            f_{x_ix_j}(\v{x}) & \dots & f_{x_ix_j}(\v{x}) \\
            f_{x_ix_j}(\v{x}) & \dots & f_{x_ix_j}(\v{x}) \\
            f_{x_ix_j}(\v{x}) & \dots & f_{x_ix_j}(\v{x}) \\
        \end{bmatrix}
    \]
    \loss{
        In generale, la matrice non é simmetrica!
    }
}

\ldef{
    Data $f:A\in\mathbb{R}^n\rightarrow\mathbb{R}$, con $A$ aperto,
    si dice che $ f \in C^2$, se \'{e} derivabile due volte e le derivate
    seconde sono continue in $A$
}

\ltheorem{ 
    Il \b{Teorema di Schwartz} dice che se $f \in C^2$, l'ordine di 
    derivazione non conta e la matrice Hessian é simmetrica. In generale
    $ f \in C^k $ le derivate di ordine $k$ non dipendono dall'ordine
    di derivazione.
}

\ltheorem{
    Il \b{teorema di Lagrange} dice che: 
    sia $f:A\in\mathbb{R}^n\rightarrow\mathbb{R}$, con $A$ aperto,
    $f \in C^1(A)$, se $\v{x}_0 \in A$, $ \v{x} \in A$, tale che il 
    segmento é in $A$, $ \exists \sigma $ appartenente al segmento
    $ \v{x}, \v{x}_0 $ tale che:
    \[
        f(\v{x}) = f(\v{x}_0) + \triangledown f (\sigma) (\v{x} - \v{x}_0)
    \]
    Che non é altro che lo sviluppo di Taylor al primo ordine con resto
    di Lagrange. 
    \ldim{
        Poniamo 
        \[
            \v{h} = \v{x} - \v{x}_0
        \]
        In modo tale che il segmento sia rappresentato da $\v{x}_0 + t\v{h}$.
        Allora
        \[
            F \colon [0,2 ] \rightarrow \mathbb{R},
            \quad \quad 
            F(t) \colon = f(\v{x} + t\v{h})
        \]
        Se abbiamo che $f \in C^1$, allora applicando Lagrange:
        \[
            F(1) - F(0) = F'(\varepsilon) (1 - 0) = F'(\varepsilon),
            \quad \text{ con } \varepsilon \in (0, 1)
        \]
        Applicando ancora Lagrange:
        \begin{gather*} 
            f(\v{x}) - f(\v{x}_0) = 
            \triangledown f(\v{x}_0 + \varepsilon \v{h} ) \cdot \v{h} =
            \\
            \sum f_{x_i} (\v{x}_0 + \varepsilon \v{h}_i) \cdot\v{h}_i 
        \end{gather*} 
        Dove $\v{x}_0 + \varepsilon \v{h}_i$ rappresenta un punto 
        del segmento.
    }
}


\ltheorem{ 
    Il \b{teorema di Lagrange} dice che: 
    sia $f:A\in\mathbb{R}^n\rightarrow\mathbb{R}$, con $A$ aperto,
    $f \in C^2(A)$, se $\v{x}_0 \in A$, $ \v{x} \in A$, tale che il 
    segmento é in $A$, $ \exists \sigma $ appartenente al segmento
    $ \v{x}, \v{x}_0 $ tale che:
    \[
        f(\v{x}) = f(\v{x}_0) + \triangledown f (\sigma) (\v{x} - \v{x}_0)
        + \frac{1}{2} ( H_f(\v{x}_0 + \varepsilon
        (\v{x} - \v{x}_0))(\v{x} - \v{x}_0)) (\v{x} - \v{x}_0)
    \]
    Che non é altro che lo sviluppo di Taylor al secondo ordine con resto
    di Lagrange. 
    \loss{
        con il resto di Peano:
        \[
            f(\v{x}_0) + \triangledown f (\sigma) (\v{x} - \v{x}_0)
            + \frac{1}{2} ( H_f(x_0 + \varepsilon
            (x - x_0))(x - x_0)) (x - x_0) +
            o(| \v{x} - \v{x}_0 |^2)
        \]
    }
    \ldim{
        Se $f \in C^2(A) \implies F \in C^2([0, 1]) $, e quindi
        \[
            F(t) = F(0) + F'(0)t + \frac{1}{2}F''(\varepsilon)t^2 \text{ con }
            \quad \varepsilon \in (0, t)
        \]
        Imponendo $t = 1$:
        \begin{eqnarray*}
            f(\v{x}) &=& f(\v{x}_0) + \triangledown f(\v{x}_0)(\v{x} - \v{x}_0) 
            + \frac{1}{2} F''(\varepsilon)
            \\
            &=& f(\v{x}_0) + \triangledown f(\v{x}_0)(\v{x} - \v{x}_0) +
            \frac{1}{2}(H_f(\v{x}_0 + \varepsilon 
            (\v{x} - \v{x}_0)) (\v{x} - \v{x}_0))(\v{x} - \v{x}_0)
        \end{eqnarray*}
    }
}

\ldef{
    Un insieme $ A \subseteq R^n$ si dice \b{connesso} se le condizioni
    \[
        \exists A_1, A_2 \text{ aperti t.c. } \quad A_1 \cup A_2 = A, \quad
        A_1 \cap A_2 = \emptyset 
    \]
    implicano che uno dei due insiemi aperti é vuoto.
}

\ltheorem {
    Dato $A \subseteq \mathbb{R}^n $ aperto e connesso, allora $\forall \v{x},
    \v{y} \in A $ esiste una curva $\varnothing \in C^1$, definita come
    $\varnothing : [0, 1] \rightarrow A$ che collega $\v{x}$ con $\v{y}$, cioé
    $\varnothing (0) = \v{x}, \varnothing (1) = \v{y}$. $A$ allora é un insieme
    \b{connesso ad archi}.
}

\lprop {
    $f:A\in\mathbb{R}^n\rightarrow\mathbb{R}$, con $A$ aperto e connesso, con
    $f$ differenziabile su $A$, se $\triangledown f = 0$ in $A$ allora $f$ é
    costante su $A$.
    \ldim{
        Fisso $x, y \in A$. Per il teorema precedente, esiste una curva
        $\varnothing$ che li connette.  Definiamo $g(t) = f(\varnothing (t)),
        \; \; t \in [0, 1]$. \\
        $g$ é derivabile perché composizione di funzioni derivabili. Ora:
        \[
            f(y) - f(x) = g(1) - g(0) = g'(\varepsilon)(1 - 0) =
            g'(\varepsilon)) = \triangledown f(\varnothing(\varepsilon)) 
            \cdot \varnothing'(\varepsilon) = 0
        \]
        E con questo abbiamo dimostrato che $f$ é sempre costante su tutto $A$.
    }
}

\ldef{
    $x_0 \in A$ si dice punto di massimo/minimo relativo (o locale) se
    \[
        f(x) \geq f(x_0) \text{ oppure } f(x) \leq f(x_0), \quad 
        \forall x \in B_\delta(x_0) \cap A
    \]
    Dove $B_\delta(x_0)$ é una palla di raggio $\delta$ di centro $x_0$.
}

\lprop{
    Se $x_0$ é punto di massimo/minimo, e punto interno di $A$, e $f$ é
    derivabile in $A$ allora $\triangledown f(x_0) = 0$
    \ldim{
        Definisco
        \[
            g_i(t) = f(x_0 + te_i), \quad \text{ dove } e_i = (0, 0, \dots , 0, 1, 0, \dots)
        \]
        Sappiamo che $g_i$ é derivabile perché lo é anche $f$, e ha
        massimo/minimo in $t = 0$.  Svolgendo:
        \[
            0 = g'(0) = \triangledown f(x_0 + 0 \cdot e_i  ) \cdot e_i = 
            \frac{\delta f}{ \delta x_i} (x_0)
        \]
        E questo vale $\forall i$; cioé, qualunque sia la componente, la
        derivata é nulla, e quindi anche il gradiente é nullo.
    }
    \loss{
        Non vale il viceversa! Se il gradiente é nullo, potrebbe anche essere
        che $x_0$ non é punto di massimo/minimo!
    }
}


\ldef{
    Se $\triangledown f(x_0) = 0$ ma $x_0$ non é punto di massimo/minimo,
    allora si dice che $x_0$ é \b{punto di sella}.
}

\ldef{
    Introduciamo la seguente notazione:
    \[
        (H\v{v}) \cdot \v{v} = (H\v{v}, \v{v}) = \sum_{i,j=1}^n h_{i,j}v_iv_j
    \]
}

\lprop {
    Se la matrice $H$ é simmetrica, allora é diagonalizzabile e ha autovalori
    $\lambda_1, \lambda_2, \dots, \lambda_n$.
}

\ldef{
    La matrice $H$ si dice:\\
    \small
    \begin{tabular}{@{}lll@{}}
    \b{Definita positiva} & se $(H\v{v}, \v{v}) > 0,  \forall \v{v}$ 
        & cioé se $\lambda_i > 0, \forall i$\\
    \b{Definita negativa} & se $(H\v{v}, \v{v}) < 0,  \forall \v{v}$ 
        & cioé se $\lambda_i < 0, \forall i$\\
    \b{Semidefinita positiva} & se $(H\v{v}, \v{v}) \geq 0,  \forall \v{v}$ 
        & cioé se $\lambda_i \geq 0, \forall i$\\
    \b{Semidefinita negativa} & se $(H\v{v}, \v{v}) \leq 0,  \forall \v{v}$ 
        & cioé se $\lambda_i \leq 0, \forall i$\\
        \multirow{2}{*}{\b{Indefinita}} & \multicolumn{2}{l}{ se 
            $\exists v,w$ t.c. $(Hv,v)>0$ e $(H\v{w}, \v{w}) < 0$, }\\
        & \multicolumn{2}{l}{cioé se 
            $\exists \lambda_1, \lambda_2$ t.c. $\lambda_1 \lambda_2 < 0$}
    \end{tabular}
}

\lprop{
    \small
    Se $H$ é definita positiva, allora 
    $(H\v{v}, \v{v}) \geq \lambda_{min} |\v{v}|^2, \forall \v{v}$.\\
    Se $H$ é definita negativa, allora 
    $(H\v{v}, \v{v}) \leq \lambda_{max} |\v{v}|^2, \forall \v{v}$.
}

\lprop{
    Se $f \in C^2$, $x_0 \in A$, $x_0$ punto interno e 
    $\triangledown f(x_0)=0$:
    \small
    Se $x_0$ é un minimo locale, allora $H_f(x_0)$ é semidefinita positiva.\\
    Se $x_0$ é un massimo locale, allora $H_f(x_0)$ é semidefinita negativa.
    \ldim{
        Manca!
    }
    Se $H_f(x_0)$ é definita positiva, allora $x_0$ é un massimo locale.\\
    Se $H_f(x_0)$ é definita negativa, allora $x_0$ é un minimo locale.
    \ldim{
        Manca!
    }
    Se $H_f(x_0)$ é indefinita, allora $x_0$ é un punto di sella.
    \ldim{
        Manca!
    }
}

\lprop{
    Per n = 2, $ A = \begin{bmatrix} a & b \\ c & d \end{bmatrix} \\
        ac - b^2 = \text{det A} = \lambda_1 \lambda_2, \quad \quad 
        a + c = \text{tr(A)} = \lambda_1 + \lambda_2$ \\
    Se detA $<  $, A è indefinita, e $\v{x}_0$ punto di sella.\\
    Se detA $=0$, sappiamo che un autovalore è 0, ma niente altro. 
    (al massimo sappiamo che A è semidefinita positiva o negativa)\\
    Se detA $>0$, e tr(A) $>0$, A è definita positivamente,
    e $\v{x}_0$ è punto di minimo.\\
    Se detA $>0$, e tr(A) $<0$, A è definita negativamente,
    e $\v{x}_0$ è punto di massimo.

    \loss{
        Se detA $ > 0 \implies ac > b^2 \implies $ $a$ e $c$ hanno lo stesso
        segno.  Quindi, per vedere il segno della traccia, mi basta osservare
        semplicemente il segno di $a$.
    }
    \loss{
        Vale anche il viceversa, cioè\\ 
        Se $\v{x}_0$ è punto di minimo, allora detA $\geq0$, tr(A) $\geq 0$\\ 
        Se $\v{x}_0$ è punto di massimo allora detA $\leq0$, tr(A) $\leq 0$.\\ 
    }
    \loss{
        La matrice A è simmetrica poichè per ipotesi $f \in C^2$.
    }
}

\ldef{
    I \b{minimi} principali \b{nord-ovest} di una matrice sono le sottomatrici
    quadrate formate a partire dall'angolo in alto a sinistra:
    \[
        A_1=\begin{bmatrix} a_{11} \end{bmatrix}, \quad
        A_2=\begin{bmatrix} a_{11}&a_{12}\\a_{21}&a_{22}  \end{bmatrix}, \quad
        A_3=\begin{bmatrix} 
            a_{11} & a_{12} & a_{13} \\
            a_{21} & a_{22} & a_{23} \\
            a_{31} & a_{32} & a_{33} \\
        \end{bmatrix}\dots
    \]
}

\lprop {
    Se A è simmetrica, sappiamo che:\\
    A è definita positiva $ \Leftrightarrow $ det$A_k > 0 $ per $k = 1 \dots n$.\\
    A è definita negativa $ \Leftrightarrow $ det$A_k (-1)^k > 0 $ per $k = 1
    \dots n$.
}

\ldef{
    Data $f:A\in\mathbb{R}^n\rightarrow\mathbb{R}$, $x_m \in A$ è punto di
    \b{minimo assoluto} per $f$ in $A$ se $f(\v{x}_m) \leq f(\v{x}), \forall
    \v{x} \in A$ e $m = f(\v{x}_m)$ è detto minimo assoluto. Uguale per il
    massimo.
}


\textbf{ciao} 


\end{multicols}
\msection{Personal reminders}
\begin{multicols}{2} 

\b{integrali}


$\sin(x)$ \'{e} una funzione \b{dispari}


$\cos(x)$ \'{e} una funzione \b{pari}
ciao io son oluca


\smallbreak


\end{multicols}

\center
\Huge{ 
\b{Rigurda gli spazi topologici.}

\b{PRIMA dell'esonero.}

}



%:set iskeyword+={,},,,(,),^,*,+,-,_,<,>,\,/,$,
    %:set iskeyword=30-200
%:set updatetime=500

%\begin{multicols}{2} 
%\end{multicols}
%\bigbreak
%\hrule
%\medbreak
%\centerline{Algoritmi:}

%\begin{multicols}{2} 

%\end{multicols}
\end{document}


















