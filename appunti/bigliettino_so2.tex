\documentclass[a4paper,10pt]{article} %default
\usepackage{geometry}
\usepackage{fontspec}
\usepackage[normalem]{ulem}
\usepackage{color}
\usepackage{listings}
%\usepackage{tikz}
%\usetikzlibrary{arrows.meta}
%\usetikzlibrary{graphs,graphdrawing}
%\usegdlibrary{force}
\geometry{top = 1mm, lmargin=2mm, rmargin=2mm, bottom=1mm}
\pagestyle{empty}
\setlength{\parindent}{0pt}

\lstset{
    basicstyle=\ttfamily,
    numberstyle=\ttfamily,
    numbers=left,
    backgroundcolor=\color[rgb]{0.9,0.9,0.9},
    columns=fullflexible,
    keepspaces=true,
    frame=lr,
    framesep=8pt,
    framerule=0pt,
    xleftmargin=20pt,
    xrightmargin=30pt,
    mathescape
}

\newcommand{\specialcell}[2][c]{%
    \begin{tabular}[#1]{@{}l@{}}#2\end{tabular}}
\newcommand{\dimo}[1]{%
    \smallbreak \par \hfill\begin{minipage}{0.92\linewidth}{ \scriptsize {\textbf{\em{Dim.}}} {#1} }\end{minipage} \smallskip \par}
\newcommand{\mtheorem}[1]{%
    {\hspace*{-10pt} \textsc {#1}}}
\newcommand{\malgorithm}[1]{%
    {\bigbreak \par \hspace*{4pt} \underline{\textbf {#1}}}}
\newcommand{\msection}[1]{%
    {\newpage\bigbreak \bigbreak \par \hfil \huge \textsc {#1}}\par}
\renewcommand{\b}[1]{%
    {\textbf{#1}}}
\renewcommand{\t}[1]{%
    {\texttt{#1}}}
\newcommand{\mdef}[1]{%
    {\smallbreak\par\begin{tabular}{ll} \textbf{Def.$\;\;$} & \begin{minipage}[t]{0.80\columnwidth}\normalsize  {#1}\end{minipage}\tabularnewline \end{tabular}}\smallskip\par}
\newcommand{\mcomment}[1]{%
    {\hfill \scriptsize {#1}}}
\newcommand{\mprop}[1]{%
    {\smallbreak\par\begin{tabular}{ll} \textbf{Prop.} & \begin{minipage}[t]{0.8\columnwidth}\emph  {#1}\end{minipage}\tabularnewline \end{tabular}}\smallskip\par}

\newcommand{\mdom}[2]{%
    {\smallbreak\par\begin{tabular}{ll} \textbf{{#1}} & \begin{minipage}[t]{0.8\columnwidth} {#2}\end{minipage}\tabularnewline \end{tabular}}\smallskip\par}

\newcommand{\pp}{%
    {\par \hspace*{0pt}}}

\newcommand{\minisection}[1]{%
    {\bigbreak \par \hfil \large \textsc {#1}} \bigbreak }

\begin{document}

%\tiny
\scriptsize

\minisection{Raccolta Domande SO2 - parte 1}

\bigskip

\mdom{Domanda 1:}{\emph{Quale delle seguenti affermazioni sulle pipe di Linux è vera?}\pp "Usando la syscall pipe, vengono automaticamente aperti 2 file descriptor"}
\smallskip

\mdom{Domanda 2:}{\emph{ Quale delle seguenti affermazioni sulle syscall di Linux che riguardano i files è falsa?} \pp "La syscall dup2(2,1) ha l'effetto di ridirigere lo stdout nello stderr"}
\smallskip

\mdom{Domanda 3:}{\emph{ Quali delle seguenti affermazione è vera? } \pp "Ad ogni filesystem corrisponde un disco fisico o parte di esso (partizione)"}
\smallskip

\mdom{Domanda 4:}{\emph{ Quale delle seguenti affermazioni sulle syscall di Linux che riguardano i files è falsa? } \pp "La syscall rename(oldpath, newpath) ha lo stesso effetto del comando bash cp oldpath newpath"}
\smallskip

\mdom{Domanda 5:}{\emph{ Quale delle seguenti affermazioni sulla syscall sigaction è vera? } \pp "Nessuna delle altre opzioni è vera"}
\smallskip

\mdom{Domanda 6:}{\emph{ Relativamente alla programmazione bash, quale delle seguenti affermazioni è vera? } \pp "Una volta dichiarato il tipo di una variabile, lo si può cambiare solo dopo aver invocato il comando unset"}
\smallskip

\mdom{Domanda 7:}{\emph{ Quale delle seguenti affermazioni sul comando ps è vera? } \pp "Senza nessun argomento, mostra tutti i processi lanciati dall'utente attuale nel terminale attuale"}
\smallskip

\mdom{Domanda 8:}{\emph{ Quale delle seguenti affermazioni sul comando kill è falsa? } \pp "Per mandare il segnale 9 al processo con PID 10, è sufficiente scrivere il comando kill -KILL \%10"}
\smallskip

\mdom{Domanda 9:}{\emph{ Si consideri il comando: find / \( -name 'Doc*' -a -type  d \) -o -newer Documenti -exec touch $'\{\}'$ ; Quale delle seguenti affermazioni è vera? } \pp "Il comando modifica tutti i tempi (atime, mtime e ctime) delle directory il cui nome comincia con Doc, e fa lo stesso anche con tutti i file e/o directory che siano stati modificati più recentemente della directory Documenti"}
\smallskip

\mdom{Domanda 10:}{\emph{ Il linuguaggio C: } \pp "È un linguaggio strutturato e compilato"}
\smallskip

\mdom{Domanda 11:}{\emph{ Si supponga di voler avere in esecuzione in background i comandi cmd1 e cmd2. Quale dei seguenti modi è corretto? } \pp "cmd1; \#premere Ctrl+Z; bg; cmd2; \#premere Ctrl+Z; bg;"}
\smallskip

\mdom{Domanda 12:}{\emph{ Quale delle seguenti affermazioni sulla comunicazione tra processi in Linux è vera? } \pp "Nessuna delle altre opzioni è vera"}
\smallskip

\mdom{Domanda 13:}{\emph{ Quale delle seguenti affermazioni sulla syscall fork è falsa? } \pp "Genera una copia esatta del processo chiamante, con alcune eccezioni; tra queste ultime vi è lo stack delle chiamate"}
\smallskip

\mdom{Domanda 14:}{\emph{ Si supponga di avere il seguente frammento di codice: int var = somefunction1(); double var2 = somefunction2(); fprintf(stdout, "\%d\textbackslash n\%lf\textbackslash n", var, var2); Quale dei seguenti frammenti di codice ha lo stesso effetto? } \pp "int var = somefunction1(); double var2 = somefunction2(); char *buf = calloc(sizeof(var2) > sizeof(var)? sizeof(var2) : sizeof(var), sizeof(char)); sprintf(buf, "\%d", var); write(1, buf, sizeof(var)); write(1, "\textbackslash n", 1); sprintf(buf, "\%lf", var2); write(1, buf, sizeof(var2)); write(1, "\textbackslash n", 1);}
\smallskip

\mdom{Domanda 15:}{\emph{ Quale delle seguenti affermazioni sulle syscall dei processi in Linux è vera? } \pp "Per qualsiasi processo è possibile conoscere il suo ambiente di esecuzione senza effettuare alcuna syscall" (dubbio)}
\smallskip

\mdom{Domanda 16:}{\emph{ Uno script sed è: } \pp "Un file composto da una sequenza di linee del tipo condizione azione. Dove la condizione è un numero di linea o una espressione regolare"}
\smallskip

\mdom{Domanda 17:}{\emph{ Quale delle seguenti affermazioni sui processi Linux è vera? } \pp "Se si vuole dare input da stdin senza redirezioni ad un processo, è necessario lanciarlo in foreground" (dubbio)}
\smallskip

\mdom{Domanda 18:}{\emph{ Si supponga di avere il seguente frammento di codice: FILE *stream = fopen(} \pp "file\_esistente.txt", "r"); fseek(stream, -100, SEEK\_END); long pos = ftell(stream); Quale dei seguenti frammenti di codice ha lo stesso effetto? "int fd = open("file\_esistente.txt", O\_RDONLY); lseek(fd, -100, SEEK\_END); long pos = lseek(fd, 0, SEEK\_CUR);" (molto dubbio)}
\smallskip

\mdom{Domanda 19:}{\emph{ Quale delle seguenti affermazioni sul comando time è falsa? } \pp "Il comando /usr/bin/time cmd può solo mostrare il tempo (di CPU, di sistema, e reale)"}
\smallskip

\mdom{Domanda 20:}{\emph{ Quale delle seguenti affermazioni è vera? } \pp "Nessuna delle altre opzioni è vera""}
\smallskip

\mdom{Domanda 21:}{\emph{ Si vuole scrivere un programma equivalente al seguente script: echo -n "Esecuzione in corso..." /bin/ls -la / echo " fatto". Quale dei seguenti frammenti di codice realizza quanto detto sopra?} \pp "Nessuna delle altre opzioni è vera" }
\smallskip

\mdom{Domanda 22:}{\emph{ Quale delle seguenti affermazioni sui comandi della bash è falsa? } \pp "Il comando type file mostra il tipo del file file (regolare, directory, etc)"}
\smallskip

\mdom{Domanda 23:}{\emph{ Quale delle seguenti affermazioni sui processi Linux è vera? } \pp "Nessuna delle altre opzioni è vera" (dubbio)}
\smallskip

\mdom{Domanda 24:}{\emph{ Quale delle seguenti affermazioni sui processi Linux è falsa? } \pp "I comandi built-in della bash generano sempre nuovi processi"}
\smallskip

\mdom{Domanda 25:}{\emph{ Quale dei seguenti campi non è presente nel process control block? } \pp "Change time"}
\smallskip

\mdom{Domanda 26:}{\emph{ Quale delle seguenti affermazioni sulle syscall dei processi in Linux è falsa?  } \pp "La syscall setuid() permette a qualsiasi processo di cambiare il suo real user ID"}
\smallskip

\mdom{Domanda 27:}{\emph{ Il linguaggio C: } \pp "È stato definito presso i laboratori di ricerca di una compagnia telefonica americana"}
\smallskip

\mdom{Domanda 28:}{\emph{ Quale delle seguenti affermazioni sulle syscall wait e waitpid è falsa? } \pp "Se una chiamata wait(\&status); ha successo, il valore di status coincide con l'exit status del processo figlio appena terminato"}
\smallskip

\mdom{Domanda 29:}{\emph{ Quale delle seguenti affermazioni sulle syscall di Linux che riguardano i files è falsa? } \pp "La syscall unlink(nomefile) rimuove sempre il contenuto di nomefile dal disco, se nomefile è un file regolare"}
\smallskip

\mdom{Domanda 30:}{\emph{ Quale dei seguenti non è un possibile stato di un processo Linux? } \pp "Continued"}
\smallskip

\mdom{Domanda 31:}{\emph{ Quale delle seguenti affermazioni sui comandi cat e od è falsa? } \pp "I comandi od file e cat file danno lo stesso risultato se file è un file di testo ASCII"}
\smallskip

\mdom{Domanda 32:}{\emph{ Quale delle seguenti affermazioni è vera? } \pp "Linux è multiutente, perché permette a più utenti contemporaneamente di essere loggati sulla stessa macchina"}
\smallskip

\mdom{Domanda 33:}{\emph{ Quale delle seguenti affermazioni sui comandi less e more è falsa? } \pp "Per terminarli occorre premere CTRL+C"}
\smallskip

\mdom{Domanda 34:}{\emph{ Quale delle seguenti affermazioni sui processi Linux è vera? } \pp "Con l'eccezione del primo processo, tutti i processi sono creati con una fork effettuata da un altro processo in esecuzione"}
\smallskip

\mdom{Domanda 35:}{\emph{ Quale delle seguenti affermazioni sulle syscall di Linux che riguardano i files è falsa? } \pp "Per richiedere un lock su un file (o su una porzione di esso), occorre chiamare la syscall ioctl" (dubbio)}
\smallskip

\mdom{Domanda 36:}{\emph{ Relativamente alla programmazione bash, quale delle seguenti affermazioni sul carattere \# è vera? } \pp "Nessuna delle opzioni è vera"}
\smallskip

\mdom{Domanda 37:}{\emph{ Si supponga di avere il seguente frammento di codice: FILE *stream = fopen(NOMEFILE, } \pp "w"); Quale dei seguenti frammenti di codice ha lo stesso effetto? "int fd = open(NOMEFILE, O\_WRONLY | O\_CREAT | O\_TRUNC, 0666);"}
\smallskip

\mdom{Domanda 38:}{\emph{ Quale dei seguenti sistemi operativi non è un antenato di Linux? } \pp "macOSX"}
\smallskip

\mdom{Domanda 39:}{\emph{ Quale delle seguenti affermazioni sui processi Linux è vera? } \pp "Eseguendo (con successo) k volte un file eseguibile, si generano k diversi processi"}
\smallskip

\mdom{Domanda 40:}{\emph{ Quale delle seguenti affermazioni sugli errori delle syscall di Linux è vera? } \pp "Per stamppe su stdout la spiegazione di un errore verificatosi in una syscall si può effettuare la seguente chiamata: printf("\%s\textbackslash n", strerror(errno));"}
\smallskip


\minisection{Raccolta Domande SO2 - ppte 2}

\mdom{Domanda 1:}{\emph{Quale delle seguenti affermazioni sulle pipe di Linux è vera?}\pp "Usando la syscall pipe, vengono automaticamente aperti 2 file descriptor"}
\smallskip

\mdom{Domanda 2:}{\emph{ Un programma scritto in un linguaggio C:}\pp "Rappresenta le stringhe come array di caratteri terminate dal carattere '\textbackslash 0'"}

\mdom{Domanda 4:}{\emph{ Quale dei seguenti linguaggi non è mai stato usato per implementare Unix? }\pp "Le altre risposte contengono tutte dei linguaggi usati per implementare Unix"}

\mdom{Domanda 6:}{\emph{ Ignorando eventuali memory leaks, quale dei seguenti frammenti di codice può portare ad un segmentation fault? } \pp "Quella dove fai un free subito prima di realloc()"} 

\mdom{Domanda 7:}{\emph{ Relativamente alla programmazione bash, quali delle seguenti affermazioni è esatta? } \pp "Nessuna delle opzioni"} 

\mdom{Domanda 8:}{\emph{ Quale delle seguenti affermazioni sui processi Linux è vera? } \pp "Per stopppe un processo in foreground, su può sia mandare un segnale SIGTSTP che premere CTRL+Z in una qualsiasi shell"} 

\mdom{Domanda 10:}{\emph{ Quale delle seguenti affermazioni sui processi Linux è falsa? } \pp "Per creare i PID dei processi si usano dei numeri interi che crescono sempre"} 

\mdom{Domanda 11:}{\emph{ Quale delle seguenti affermazioni sulle syscall di Linux che riguardano le directory è falsa? } \pp "Per poter cambiare il contenuto di una directory occorre aprirla con la syscall opendir" (molto in dubbio)} 

\mdom{Domanda 15:}{\emph{ Per modificare tutte le occorrenze della lettera o ed i rispettivamente in O ed I di un file di testo, quale comando è più appropriato utilizzare? } \pp "tr"} 

\mdom{Domanda 16:}{\emph{ Si supponga di voler vedere, per tutti i processi dell'utente utente, il suo PID, il suo PPID, il comando usato per lanciare il processo (con tutti gli argomenti), la usa occupazione totale di memoria in kB e la sua attuale occupazione di memoria in RAM (senza considerare quindi la ppte eventualmente swappata su disco), sempre in kB. Quale dei seguenti comandi è quello corretto? } \pp "ps -uutente -o pid,ppid,cmd,vsz,rss"} 

\mdom{Domanda 17:}{\emph{ Una espressione regolare: } \pp "Descrive implicitamente un insieme di stringhe che hanno almeno un match con se stessa"} 

\mdom{Domanda 20:}{\emph{ Quale delle seguenti affermazioni è vera? } \pp "Ogni risorsa di un sistema Unix, ad eccezione dei processi e periferiche hardware, è rappresentato da un file"} 

\mdom{Domanda 21:}{\emph{ Quale delle seguenti affermazioni sulle funzioni malloc, calloc, realloc e free è falsa? } \pp "Le due chiamate calloc(N, sizeof(int)) e realloc(NULL, N*sizeof(int)) hanno sempre lo stesso effetto"} 

\mdom{Domanda 24:}{\emph{Quale delle seguenti affermazioni sulle syscall di Linux che riguardano i files e che si trovano nella sezione 2 del manuale è falsa?} \pp "Le syscall Linux permettono solamente le seguenti operazioni: apertura, chiusura, scrittura, lettura, posizionamento"} 

\mdom{Domanda 25:}{\emph{ Quale delle seguenti affermazioni sui processi Linux è vera? } \pp "Nessuna delle altre opzioni è vera"} 

\mdom{Domanda 26:}{\emph{ Si consideri il comando: find / \( -name 'Doc*' -a -type d \) -o -newer Documenti -exec touch $'R\{\}'$ \\; : } \pp "Nessuna delle altre opzioni è vera" (dubbio)} 

\mdom{Domanda 27:}{\emph{ Relativamente alla programmazione bash, quale delle seguenti affermazioni è vera? } \pp "Nessuna delle altre opzioni è vera"} 

\mdom{Domanda 31:}{\emph{ Quale delle seguenti affermazioni sul comando kill è vera? } \pp "Può essere usato per ottenere lo stesso risultato tanto del CTRL+C quanto del CTRL+Z"} 

\mdom{Domanda 33:}{\emph{ Quale delle seguenti affermazioni sui comandi cmp, diff e patch è vera? } \pp "E' possibile usare il comando patch solo se si ha l'output del comando diff"} 

\mdom{Domanda 34:}{\emph{ Si supponga di avere il seguente frammento di codice: FILE *stream = fopen( "file\_esistente.txt", "r"); int var; double var2; fscanf(stream, "\%d\textbackslash n\%lf\textbackslash n", \&var, \&var2); e che il file file\_esistente.txt abbia il seguente contenuto: 4567 34.56. Quale dei seguenti frammenti di codice ha lo stesso effetto? } \pp "int fd = open("file\_esistente.txt", O\_RDONLY); int var; double var2; char buf[4]; read(fd, buf, sizeof(var)); var = atoi(buf); read(fd, buf, sizeof(var2)); var2 = atof(buf);" }

\mdom{Domanda 36:}{\emph{ Si supponga che sia appena stata eseguita la seguente riga di codice di un processo: int pid = fork(); quale delle seguenti affermazioni è vera? } \pp "C'è un nuovo processo pronto per andare in esecuzione, a meno che la variabile pid non valga -1"} 

\mdom{Domanda 37:}{\emph{ Quale delle seguenti affermazioni è vera? } \pp "Richard Stallman ha descritto per primo la licenza GPL"} 


\minisection{1665528.html}
\bigskip

\mdom{Domanda 4:}{\emph{ Quali dei seguenti frammenti di codice è corretto? } \pp "int pid = fork(); if (pid == 0) \{ /* fai qualcosa, sei il figlio */\} else if (pid > 0) \{ /* fai qualcosa, sei il padre */\} else \{ perror("fork failed"); \}"}
\smallskip

\mdom{Domanda 10:}{\emph{ Relativamente alla programmazione bash, la variabilee IFS: } \pp "Rappresenta una variabile contenente la sequenza di tutti i caratteri utilizzati per la separazione in parole (word splitting)"}
\smallskip

\mdom{Domanda 16:}{\emph{ Una directory di un filesystem: } \pp "Ha sempre una directory padre, eventualmente corrisponde a se stessa" (dubbio)}
\smallskip

\mdom{Domanda 17:}{\emph{ Si supponga di voler lanciare in background i comandi cmd1 e cmd2. Quale dei seguenti modi è corretto? } \pp "cmd1 \& cmd2 \&"}
\smallskip

\mdom{Domanda 20:}{\emph{ Un programma scritto in linguaggio C: } \pp "Rappresenta le stringhe come array di caratteri terminate dal carattere '\textbackslash0'"}
\smallskip

\mdom{Domanda 25:}{\emph{ Quale delle seguenti affermazioni sui comandi della bash è vera? } \pp "Eseguendo il comando echo `date` viene stampata la data e l'ora corrente (secondo l'orologio di sistema)"}
\smallskip

\mdom{Domanda 26:}{\emph{ Quale delle seguenti affermazioni sui processi Linux è falsa? } \pp "Digitare un comando sulla shell genera sempre un nuovo processo"}
\smallskip

\mdom{Domanda 31:}{\emph{ Per eliminare tutte le linee duplicate in un file di testo (senza preoccuppsi dell'ordinamento delle righe) occorre: } \pp "utilizzare il comando uniq con opzione -u"}
\smallskip

\mdom{Domanda 32:}{\emph{ La stringa Informatica9000 ha un match con la seguente REGEX: } \pp "Informatica9000\$"}
\smallskip

\mdom{Domanda 33:}{\emph{ Quale delle seguenti affermazioni sui processi Linux è vera? } \pp "Nessuna delle altre opzioni è vera"}
\smallskip

\mdom{Domanda 35:}{\emph{ Quale delle seguenti affermazioni sui segnali Linux è vera? } \pp "È possibile per un qualunque processo inviare un segnale ad un qualsiasi altro processo dello stesso utente"}
\smallskip

\mdom{Domanda 37:}{\emph{ Si consideri il comando: find Doc* \( -name 'Doc*' -a -type d \) -o -newer Documenti -exec touch '\{\}' ; Assumendo di eseguirlo in una directory dove esistano delle sottodirectory Documents e Documenti, quale delle seguenti affermazioni è vera? } \pp "Nessuna delle altre opzioni è vera"}
\smallskip








\smallskip





\end{document}
